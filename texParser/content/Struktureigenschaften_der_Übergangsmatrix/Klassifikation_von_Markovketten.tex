\textbf{Definition 2.1}[erreichbar, kommunizieren, wesentlich]
\mbox{}
\begin{itemize}
\item[a)] Ein Zustand $y \in E$ heißt erreichbar von $x \in E$ $(x \rightarrow y)$, falls ein $n \in \mathbb{N}_{0}$ existiert mit $p_{n}(x,y)>0$
\item[b)] Die Zustände $x,y \in E$ kommunizieren $(x\leftrightarrow y)$, falls $x \rightarrow y$ und $y \rightarrow x$. 
\item[c)] Eine nichtleere Teilmenge $\emptyset \neq K \subseteq E$ heißt kommunizierende Klasse, falls
\begin{itemize}
\item[(i)] $x \leftrightarrow y$ für alle $x,y \in K$
\item[(ii)] aus $x \in K$ und $y \in E$ mit $x \leftarrow y$ folgt $y \in K$ $\quad (Abgeschlossenheit)$
\end{itemize}
\item[d)] Ist $x \in E$ Element einer kommunizierenden Klasse, so heißt x wesentlich(sonst unwesentlich).
\end{itemize}

\textbf{Bemerkung 2.16}
Jedes $x \in E$ liegt höchstens in einer kommunizierenden Klasse. 

\textbf{Beispiel 2.13}
\mbox{}
\begin{itemize}
\item kommunizierende Klassen: $\lbrace 1,2,3 \rbrace$, $\lbrace 5 \rbrace$
\item unwesentliche Zustände: $\lbrace 4,6,7 \rbrace$
\end{itemize}
\begin{figure}[H].
\centering
\includegraphics[scale=0.4]{Beispiel_Kommunizierende_Klasse}
\caption{Markovkette mit zwei kommunizierenden Klassen}
\end{figure}
\noindent

\textbf{Satz 2.14}
Die Relation $\leftrightarrow$ ist eine Äquivalenzrelation. Auf der Menge der wesentlichen Zustände sind die zugehörigen Äquivalenzklassen die kommunizierenden Klassen.

\textbf{Beweis 2.28}
Offensichtlich ist die Relation $\leftrightarrow$ symmetrisch und reflexiv. Seien nun $x,y,z \in E$ mit $x \leftrightarrow y$ und $y \leftrightarrow z$. Dann gibt es $n_{1},n_{2} \in \mathbb{N}$ mit $p_{n_{1}}(x,y)>0$ und $p_{n_{2}}(y,z)>0$.
\\
Aus der Chapman-Kolmogorov-Gleichung folgt
\begin{equation*}
p_{n_{1} + n_{2}}(x,z) \geq p_{n_{1}}(x,y) \cdot p_{n_{2}}(y,z) > 0 \Rightarrow x \rightarrow z
\end{equation*}
Analog ergibt sich $z \rightarrow x$. Somit ist $\leftrightarrow$ auch transitiv. Die zweite Aussage folgt direkt aus der Definition der kommunizierenden Klasse.

\textbf{Satz 2.14}
\label{rekkurent und x -> y so gilt y -> x und y rekurrent}
Sei $(X_{n})_{n \in \mathbb{N}_{0}}$ eine $(\nu,P)$-Markovkette mit Zustandsraum E. Wenn $x \in E$ rekurrent ist und $x \rightarrow y$, so gilt $y \rightarrow x$ und y ist rekurrent.  

\textbf{Beweis 2.28}
\dashuline{zu zeigen}: $y \rightarrow x$
\\
\\
Angenommen $y \not\rightarrow x$, d.h. $p_{n}(y,x) = 0$ für alle  $n \in \mathbb{N}$. Wähle $n_{0} \in \mathbb{N}_{0}$ so, dass
\begin{equation*}
p_{n_{0}}(x,y) > 0.
\end{equation*}
Da x rekurrent ist, gilt
\begin{equation*}
0 \stackrel{\mathrm{Satz} \: \ref{alternative Chrakterisierung von rekurrent/transient}}{=} \mathbb{P}_{x}[X_{n} = x \: für \: endlich \: viele \: n \in \mathbb{N}_{0}]
\end{equation*}
\begin{equation*}
\geq \mathbb{P}_{x}[X_{n_{0}} = y, X_{n_{0} + 1} \neq x, X_{n_{0} + 2} \neq x,...]
\end{equation*}
\begin{equation*}
= \mathbb{P}_{x}[X_{n_{0}}=y] \cdot \mathbb{P}_{x}[X_{n_{0}} = y,  X_{n_{0} + 1} \neq x, X_{n_{0} + 2} \neq x,... \: | \: X_{n_{0}} = y]
\end{equation*}
\begin{equation*}
\stackrel{\mathrm{Satz} \: \ref{vorangegangene und zukünftige Ereignisse}}{=} p_{n_{0}}(x,y) \cdot \mathbb{P}_{y}[X_{1} \neq x, X_{2} \neq x,...]
\end{equation*}
Setze $A_{n} = \lbrace X_{1} \neq x,...,X_{n} \neq x \rbrace.$ Dann gilt 
\begin{equation*}
A_{n} \downarrow \bigcap_{k=1}^{\infty} A_{k} = \lbrace X_{1} \neq x, X_{2} \neq x,...\rbrace 
\end{equation*}
und
\begin{equation*}
\mathbb{P}_{y}[X_{1} \neq x,..., X_{n} = x] = 1 - \mathbb{P}_{y}[{\lbrace X_{1} \neq x,..., X_{n} = x \rbrace}^{C}] \geq 1 - \sum_{k =1}^{n} \underbrace{\mathbb{P}_{y}[X_{k} = x]}_{p_{n}(y,x) = 0} = 1
\end{equation*}
Damit erhält man aus der Stetigkeit des Wahrscheinlichkeitsmaßes $\mathbb{P}_{y}$
\begin{equation*}
\mathbb{P}_{y}[X_{1} \neq x, X_{2} \neq x,...] = \lim_{n \to \infty} \mathbb{P}_{y}[X_{1} \neq x,...,X_{n} \neq x] = 1.
\end{equation*}
Also
\begin{equation*}
0 \geq p_{n_{0}}(x,y) \cdot \mathbb{P}_{y}[X_{1} \neq x, X_{2} \neq x,...] = p_{n_{0}}(x,y) > 0 \: \: \lightning
\end{equation*}
Folglich gilt $y \rightarrow x$.

\noindent
\dashuline{zu zeigen}: y ist rekurrent
\\
\\
Seien nun $k,l \in \mathbb{N}$ so gewählt, dass $p_{k}(x,y) > 0$ und $p_{l}(y,x) > 0$. Dann ergibt sich aus der Chapman-Kolmogorov-Gleichung
\begin{equation*}
 p_{k+l+n}(y,y) \geq p_{l}(y,x) \cdot p_{n}(x,x) \cdot p_{k}(x,y) \qquad \forall n \in \mathbb{N}
\end{equation*} 
Also
\begin{equation*}
\sum_{n=1}^{\infty} p_{k+l+n}(y,y) \geq \underbrace{p_{l}(y,x)}_{>0} \cdot \underbrace{p_{k}(x,y)}_{>0} \cdot \sum_{n=1}^{\infty} p_{n}(x,x) \stackrel{\mathrm{Satz} \: \ref{alternative Chrakterisierung von rekurrent/transient}}{=} \infty.
\end{equation*}
da x rekurrent ist.
\textbf{Korollar 2.8} 
Rekurrente Zustände sind wesentlich.

\textbf{Beweis 2.28}
Sei $x \in E$ rekurrent, und setze $K(x) := \lbrace y \in E \: : \: x \rightarrow y \rbrace$. Nach Satz $\ref{rekkurent und x -> y so gilt y -> x und y rekurrent}$ gilt aber $y \rightarrow x$ für alle $y \in K(x)$. Folglich ist $K(x)$ eine kommunizierende Klasse, d.h. x ist wesentlich.
 
\textbf{Bemerkung 2.16}
Unwesentliche Zustände sind transient.

\textbf{Satz 2.14}
\label{x und y selbe Periode, x transient y auch x nullrekurrent y auch}
Sei $(X_{n})_{n \in \mathbb{N}_{0}}$ eine $(\nu,P)$-Markovkette mit Zustandsraum E und $x,y \in E$. Wenn $x \leftrightarrow y$, so gilt
\begin{itemize}
\item[a)] x und y haben die selbe Periode, d.h. $d(x) = d(y)$ 
\item[b)] x ist transient $\Leftrightarrow$ y ist transient
\item[c)] x ist nullrekurrent $\Leftrightarrow$ y ist nullrekurrent
\end{itemize}

\textbf{Bemerkung 2.16}
\label{Bemerkung 16}
Ist $x \in E$ positiv rekurrent und gilt $x \rightarrow y$, so ist auch y positiv rekurrent.

\textbf{Beweis 2.28}
a) \dashuline{zu zeigen}: $x \leftrightarrow y \quad \Rightarrow \quad d(x) = d(y)$
\\
\\
Bezeichne mit $D(x) := \lbrace n \in \mathbb{N}_{0} \: : \: p_{n}(x,x)>0 \rbrace$, $x \in E$. Seien nun $x,y \in E$ mit x $\leftrightarrow$ y. Wähle $m,n \in \mathbb{N}_{0}$ so, dass $p_{m}(x,y)>0$ und $p_{n}(y,x)>0$. Dann gilt für jedes $k \in D(y)$ aufgrund der Chapman-Kolmogorov-Gleichung
\begin{equation*}
p_{m+k+n}(x,x) \geq p_{m}(x,y) \cdot p_{k}(y,y) \cdot p_{n}(y,x) > 0
\end{equation*}
Folglich ist $m+k+n \in D(x)$. Ist nun d ein Teiler von D(x), so gilt
\begin{equation*}
 d \: | \: \lbrace m+k+n \: : \: k \in D(y) \rbrace
\end{equation*} 
Da aber $m+n \in D(x)$ und somit $d \: | \: (m+n)$, folgt $d \: | \: D(y)$. Also, 
\begin{equation*}
d(x) = ggT(D(x)) \leq ggT(D(y)) = d(y)
\end{equation*}
Durch Vertauschen der Rollen von x und y folgt analog $d(y) \leq d(x)$. Also $d(x) = d(y)$.
\\
\\
b) $"\Rightarrow"$ Sei x transient. Da x $\leftrightarrow$ y, existieren $k,l \in \mathbb{N}$ so, dass  $p_{k}(x,y)>0$ und $p_{l}(y,x)>0$. Dann folgt aus der Chapman-Kolmogorov-Gleichung
\begin{equation*}
p_{k+l+n}(x,x) \geq p_{k}(x,y) \cdot p_{n}(y,y) \cdot p_{l}(y,x) > 0 \quad \forall n \in \mathbb{N}.
\end{equation*}
Daraus folgt aus Satz $\ref{alternative Chrakterisierung von rekurrent/transient}$ 
\begin{equation*}
\infty > \sum_{n=1}^{\infty} p_{k+l+n}(x,x) \geq \underbrace{p_{k}(x,y)}_{>0} \cdot \underbrace{p_{l}(y,x)}_{>0} \sum_{n=1}^{\infty} p_{n}(y,y) \quad \Rightarrow \quad \sum_{n=1}^{\infty} p_{n}(y,y) < \infty
\end{equation*}
Also ist y transient.
\\
$"\Leftarrow"$ Analog.
\\
\\
c) \: $"\Rightarrow"$ Sei x nullrekurrent. Da $x \leftrightarrow y$ existieren somit $k,l \in \mathbb{N}$ mit $p_{k}(x,y) > 0$ und $p_{l}(y,x) > 0$. Wiederum folgt aus der Chapman-Kolmogorov-Gleichung
\begin{equation*}
p_{k+l+n}(x,x) \geq p_{k}(x,y) \cdot p_{n}(y,y) \cdot p_{l}(y,x) \quad \forall n \in \mathbb{N}
\end{equation*}
Dann folgt aus Satz $\ref{nullrekurrent und limes}$
\begin{equation*}
0 = \lim_{n \to \infty}p_{k+l+n}(x,x) \geq \limsup_{n \to \infty} \underbrace{p_{k}(x,y)}_{>0} \cdot \underbrace{p_{l}(y,x)}_{>0} \cdot p_{n}(y,y) \quad \Rightarrow \quad \lim_{n \to \infty} p_{n}(y,y) = 0
\end{equation*}
Folglich ist y nach Satz $\ref{nullrekurrent und limes}$ nullrekurrent.
\\
$"\Leftarrow"$ Analog.

\textbf{Definition 2.1}[irreduzibel]
Eine stochastische Matrix P auf E heißt irreduzibel, falls E nur aus einer kommunizierenden Klasse besteht. Eine $(\nu,P)$-Markovkette $(X_{n})_{n \in \mathbb{N}_{0}}$ heißt irreduzibel, falls P irreduzibel ist.

\textbf{Satz 2.14}
\label{irr. Markovkette x positiv rekurrent}
Ist $(X_{n})_{n \in \mathbb{N}_{0}}$ eine irreduzible $(\nu,P)$-Markovkette auf einem endlichen Zustandsraum E, so ist $x \in E$ positiv rekurrent.

\textbf{Beweis 2.28}
Zunächst einmal gilt für jedes $x \in E$
\begin{equation*}
\sum_{y \in E}G(x,y) = \sum_{n=0}^{\infty} \sum_{y \in E} p_{n}(x,y) = \sum_{n=0}^{\infty} 1 = \infty
\end{equation*}
Da E endlich ist, gibt es folglich ein $y \in E$ mit $G(x,y) = \infty$. Da E aufgrund der Irreduzibilität nur aus einer kommunizierenden Klasse besteht, ist insbesondere $y \rightarrow x$. Folglich existiert ein $m \in \mathbb{N}$ mit $p_{m}(x,y)>0$. Aus der Chapman-Kolmogorov-Gleichung folgt zudem $p_{m+n}(x,x) \geq p_{n}(x,y) \cdot p_{m}(y,x)$. Also,
\begin{equation*}
G(x,x) \geq \sum_{n=0}^{\infty} p_{n}(x,y)p_{m}(y,x) = \underbrace{p_{m}(y,x)}_{>0} \cdot G(x,y) = \infty
\end{equation*}
Somit ist x nach Satz $\ref{alternative Chrakterisierung von rekurrent/transient}$ rekurrent. Aus Satz $\ref{rekkurent und x -> y so gilt y -> x und y rekurrent}$ folgt dann aber, dass jeder Zustand in E rekurrent ist. Angenommen $x \in E$ wäre nullrekurrent. Dann folgt aus Satz $\ref{x und y selbe Periode, x transient y auch x nullrekurrent y auch}$, dass jeder Zustand nullrekurrent ist. Aber dann folgt aus Korollar $\ref{nullrekurrent und limes}$
\begin{equation*}
1 = \lim_{n \to \infty} \sum_{y \in E} p_{n}(x,y) \stackrel{\vert E \vert < \infty}{=} 0 \: \: \lightning 
\end{equation*}
$\Rightarrow$ alle Zustände in E sind positiv rekurrent.  

\mbox{}
\\
\textbf{Satz 2.14}
\label{irreduzibel, y rekurrent -> Px=1 , y transient -> Px<1 }
Ist $(X_{n})_{n \in \mathbb{N}_{0}}$ eine irreduzible $(\nu,P)$-Markovkette mit Zustandsraum E. Dann gilt
\begin{itemize}
\item[a)]$y \in E$ ist rekurrent $\quad$ $\Rightarrow$ $\quad$ $\mathbb{P}_{x}[S_{\lbrace y \rbrace} < \infty] = 1$ $\quad$ $\forall x,y \in E$ 
\item[b)]$y \in E$ ist transient $\quad$ $\Rightarrow$ $\quad$ $\mathbb{P}_{x}[S_{\lbrace y \rbrace} < \infty] < 1$ $\quad$ $\forall x,y \in E$  
\end{itemize}

\textbf{Beweis 2.28}
Da $(X_{n})_{n \in \mathbb{N}_{0}}$ irreduzibel ist, folgt $x \leftrightarrow y$ für alle $x,y \in E$. Insbesondere sind nach Satz $\ref{rekkurent und x -> y so gilt y -> x und y rekurrent}$ und $\ref{x und y selbe Periode, x transient y auch x nullrekurrent y auch}$ alle Zustände entweder rekurrent oder transient, d.h.
\begin{itemize}
\item[a)] $\mathbb{P}_{y}[S_{\lbrace y \rbrace} < \infty] = 1 \: \forall y \in E$
\item[b)] $\mathbb{P}_{y}[S_{\lbrace y \rbrace} < \infty] < 1 \: \forall y \in E$
\end{itemize}
Sei nun $x,y \in E$ mit $x \neq y$. Dann existieren wegen $x \leftrightarrow y$ ein $n \in \mathbb{N}$ so, dass
\begin{equation*}
n = \min \lbrace k \in \mathbb{N} \: : \: p_{k}(y,x)>0 \rbrace.
\end{equation*}
Dann gilt für jedes N>n
\begin{equation*}
\mathbb{P}_{y}[S_{\lbrace y \rbrace} \leq N, X_{n} = x]
\end{equation*}
\begin{equation*}
= \sum_{k=1}^{N} \mathbb{P}_{y}[S_{\lbrace y \rbrace} = k, X_{n} = x]
\end{equation*}
\begin{equation*}
= \sum_{k=1}^{n-1} \mathbb{P}_{y}[S_{\lbrace y \rbrace} = k]\mathbb{P}_{y}[X_{n} = x \: | \: X_{S_{\lbrace y \rbrace}} = y, S_{\lbrace y \rbrace} = k] + \sum_{k=n+1}^{N} \mathbb{P}_{y}[X_{n} =x]\mathbb{P}_{y}[S_{\lbrace y \rbrace} = k \: | \: X_{n} = x]
\end{equation*}
\begin{equation*}
= \sum_{k=1}^{n-1} \mathbb{P}_{y}[S_{\lbrace y \rbrace} = k]\mathbb{P}_{y}[X_{n-k} = x] + \sum_{k=n+1}^{N} \mathbb{P}_{y}[X_{n} =x]\mathbb{P}_{y}[S_{\lbrace y \rbrace} = k-n]
\end{equation*}
wobei im letzten Schritt sowohl die Markoveigenschaft als auch die starke Markoveigenschaft benutzt wurde. Zudem gilt nach Wahl von n, dass
\begin{equation*}
\mathbb{P}_{y}[X_{n-k} = x] = 0 \quad \forall \; k \in \lbrace 1,2,...,n-1 \rbrace
\end{equation*}
Daraus folgt
\begin{equation*}
\mathbb{P}_{y}[S_{\lbrace y \rbrace} < \infty, X_{n} = x]
\end{equation*}
\begin{equation*}
= \lim_{N \to \infty} \mathbb{P}_{y}[S_{\lbrace y \rbrace} \leq N, X_{n} = x]
\end{equation*}
\begin{equation*}
= p_{n}(y,x) \sum_{k=n+1}^{\infty}\mathbb{P}_{x}[S_{\lbrace y \rbrace} = k-n]
\end{equation*}
\begin{equation*}
 = p_{n}(y,x) \cdot \mathbb{P}_{x}[S_{\lbrace y \rbrace} < \infty]
\end{equation*}
\begin{itemize}
\item[a)] Ist nun $\mathbb{P}_{y}[S_{\lbrace y \rbrace} < \infty] = 1$, so folgt
\begin{equation*}
p_{n}(y,x) = \mathbb{P}_{y}[S_{\lbrace y \rbrace} < \infty, X_{n} = x] = p_{n}(y,x) \cdot \mathbb{P}_{x}[S_{\lbrace y \rbrace} < \infty] \: \Leftrightarrow \: \mathbb{P}_{x}[S_{\lbrace y \rbrace} < \infty] = 1 
\end{equation*}
\item[b)] Ist nun $\mathbb{P}_{y}[S_{\lbrace y \rbrace} < \infty] < 1$, so gilt
\begin{equation*}
p_{n}(y,x) > \mathbb{P}_{y}[S_{\lbrace y \rbrace} < \infty, X_{n} = x] = p_{n}(y,x) \cdot \mathbb{P}_{x}[S_{\lbrace y \rbrace} < \infty] \: \Leftrightarrow \: \mathbb{P}_{x}[S_{\lbrace y \rbrace} < \infty] < 1 
\end{equation*}
\end{itemize}

\textbf{Definition 2.1}[Rekurrenz/Transienz einer Markovkette]
Eine irreduzible $(\nu,P)$-Markovkette heißt rekurrent/transient, wenn ein Zustand rekurrent/transient ist.
