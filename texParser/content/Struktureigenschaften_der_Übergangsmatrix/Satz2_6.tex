\label{alternative Chrakterisierung von rekurrent/transient}
Sei $(X_{n})_{n \in \mathbb{N}_{0}}$ eine $(\nu,P)$-Markovkette mit Zustandsraum E. Dann gilt
\begin{itemize}
\item[a)] x ist rekurrent $\Leftrightarrow$ $\mathbb{P}_{x}[X_{n} = x \: \: u.o.] = 1$ $\Leftrightarrow$ $\sum_{n=1}^{\infty} p_{n}(x,x) = \infty$
\item[b)] x ist transient $\Leftrightarrow$ $\mathbb{P}_{x}[X_{n} = x \: \: u.o.] = 0$ $\Leftrightarrow$ $\sum_{n=1}^{\infty} p_{n}(x,x) < \infty$
\end{itemize}
Insbesondere ist jeder Zustand $x \in E$ entweder rekurrent oder transient.