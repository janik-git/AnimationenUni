\mbox{}
\begin{itemize}
\item[a)] Offensichtlich gilt für die Stoppzeit $S_{A} \wedge n$, dass $\mathbb{E}_{x}[S_{A} \wedge n] \leq n$ für alle $x \in E$ und $n \in \mathbb{N}$. Aus der Dynkin-Formel(Satz $\ref{Dynkin-Formel}$) angewendet auf $T = S_{A} \wedge n$ und $f = h \wedge n$ folgt zusammen mit dem Lemma von Fatou
\begin{equation*}
h(y) = \liminf_{m \to \infty} h(y) \wedge m \geq \liminf_{m \to \infty}
\liminf_{n \to \infty} \mathbb{E}_{y}[h(X_{S_{A} \wedge n}) \wedge m] 
\end{equation*}
\begin{equation*}
\geq \mathbb{E}_{y}[h(X_{S_{A}})\mathbbm{1}_{S_{A} < \infty}] \geq \inf_{z \in A} h(z) \mathbb{P}_{y}[S_{A} < \infty]
\end{equation*}
Also,
\begin{equation*}
\mathbb{P}_{y}[S_{A}< \infty] \leq \dfrac{h(y)}{\inf_{z \in A} h(z)} < 1
\end{equation*}
Zusammen mit Lemma $\ref{Px[SA < unendlich] < 1, so y transient}$ a) folgt somit, dass jeder Zustand in E transient ist.
\item[b)] \dashuline{zu zeigen}: Für jedes $c \geq 0$ gilt $\mathbb{P}_{x}[S_{\lbrace h > c \rbrace} < \infty] = 1 \quad \forall x \in \lbrace h \leq c \rbrace$
\\
\\
Aus der Irreduziblität folgt zunächst einmal, dass für jedes $c \geq 0$
\begin{equation*}
\mathbb{P}_{x}[S_{\lbrace h > c \rbrace} = \infty] < 1 \qquad \forall x \in \lbrace h \leq c \rbrace
\end{equation*}
Folglich existiert zu jedem $x \in \lbrace h \leq c \rbrace$ ein $N_{x} \in \mathbb{N}$ und $\epsilon_{x} > 0$ so, dass
\begin{equation*}
\mathbb{P}_{x}[S_{\lbrace h > c \rbrace} > n] \leq 1 - \epsilon_{x} \qquad n \geq N_{x}
\end{equation*}
Setze $N := \max {\lbrace N_{x} \: : \: x \in {\lbrace h \leq c \rbrace}  \rbrace}$ und $\epsilon := \min \lbrace \epsilon_{x} \: : \: x \in \lbrace h \leq c \rbrace \rbrace$. Da nach Voraussetzungen die Menge $\lbrace h \leq c \rbrace$ endlich ist für jedes $c \geq 0$, folgt $N < \infty$, $\epsilon > 0$ und
\begin{equation*}
\mathbb{P}_{x}[S_{\lbrace h > c \rbrace} > n] \leq 1 - \epsilon \quad \forall n \geq N \quad \mathrm{und} \quad x \in {\lbrace h \leq x \rbrace}
\end{equation*}
Weiterhin gilt für jedes $k \in \mathbb{N}$
\begin{equation*}
\mathbb{P}_{x}[S_{\lbrace h > c \rbrace}> kN]
\end{equation*}
\begin{equation*}
= \sum_{y \in {\lbrace h \leq c \rbrace}} \mathbb{P}_{x}[S_{\lbrace h > c \rbrace}> kN \: | \: S_{\lbrace h > c \rbrace} > (k-1)N, X_{(k-1)N}=y] \mathbb{P}_{x}[S_{\lbrace h > c \rbrace} > (k-1)N, X_{(k-1)N}=y]
\end{equation*}
\begin{equation*}
\stackrel{\mathrm{Satz} \: \ref{vorangegangene und zukünftige Ereignisse}}{=} \sum_{y \in {\lbrace h \leq c \rbrace}} \underbrace{\mathbb{P}_{y}[S_{\lbrace h > c \rbrace}> N]}_{\leq 1 - \epsilon} \mathbb{P}_{x}[S_{\lbrace h > c \rbrace} > (k-1)N, X_{(k-1)N}=y]
\end{equation*}
\begin{equation*}
\leq (1-\epsilon) \mathbb{P}_{x}[S_{\lbrace h > c \rbrace} > (k-1)N]
\end{equation*}
Folglich ergibt sich induktiv, dass $\mathbb{P}_{x}[S_{\lbrace h > c \rbrace} > (k-1)N] \leq (1-\epsilon)^{k}$ für alle $k \in \mathbb{N}$ und $x \in \lbrace h \leq c \rbrace$.
\\
Somit erhält man 
\begin{equation*}
\mathbb{P}_{x}[S_{\lbrace h > c \rbrace} = \infty] = \limsup_{k \to \infty} \mathbb{P}_{x}[S_{\lbrace h > c \rbrace} = kN] \leq \limsup_{k \to \infty} (1-\epsilon)^{k} = 0 \quad \forall x \in {\lbrace h > c \rbrace}
\end{equation*}
\dashuline{zu zeigen}: $\mathbb{P}_{x}[S_{\lbrace h > c \rbrace} < \infty] = 1 \quad \forall x \in A$
\\
\\
Für jedes $c \geq 0$ gilt für die Stoppzeit $S_{A} \wedge n \wedge S_{{\lbrace h > c \rbrace}}$, dass $\mathbb{E}_{x}[S_{A} \wedge n \wedge S_{{\lbrace h > c \rbrace}}] \leq n < \infty$ für alle $n \in \mathbb{N}$ und $x \in E$. Mit der Dynkin-Formel (Satz $\ref{Dynkin-Formel}$) angewendet auf $T=S_{A} \wedge n \wedge S_{{\lbrace h > c \rbrace}}$ und $f = h \wedge m$ folgt zusammen mit dem Lemma von Fatou
\begin{equation*}
h(x) = \liminf_{m \to \infty} h(x) \wedge m
\end{equation*} 
\begin{equation*}
\geq \liminf_{m \to \infty} \liminf_{n \to \infty} \mathbb{E}_{x}[h(X_{S_{A} \wedge n \wedge S_{\lbrace h > c \rbrace}} ) \wedge m] \geq \mathbb{E}_{x}[h(X_{S_{A} \wedge n \wedge S_{\lbrace h > c \rbrace}} )]
\end{equation*}
Daraus folgt
\begin{equation*}
h(x) \geq \mathbb{E}_{x}[h(X_{S_{{\lbrace h > c \rbrace}}}) \mathbbm{1}_{S_{{\lbrace h > c \rbrace} < \infty}}\mathbbm{1}_{S_{{\lbrace A \rbrace} = \infty}}] \geq c \cdot \mathbb{P}_{x}[S_{{\lbrace h > c \rbrace}} < \infty, S_{{\lbrace h > c \rbrace}} = \infty]
\end{equation*}
\begin{equation*}
= c \cdot \mathbb{P}_{x}[S_{A} = \infty]
\end{equation*}
Da dies für jedes $c \geq 0$ gilt, folgt schließlich
\begin{equation*}
\mathbb{P}_{x}[S_{A} = \infty] \leq \limsup_{c \to \infty} \dfrac{h(x)}{c} = 0 \quad \Leftrightarrow \quad \mathbb{P}_{x}[S_{A} < \infty] = 1 \qquad \forall x \in E
\end{equation*}
Folglich ist nach Lemma $\ref{Px[SA < unendlich] < 1, so y transient}$ b) jeder Zustand $y \in E$ rekurrent.
\end{itemize}\mbox{}
\begin{itemize}
\item[a)] Offensichtlich gilt für die Stoppzeit $S_{A} \wedge n$, dass $\mathbb{E}_{x}[S_{A} \wedge n] \leq n$ für alle $x \in E$ und $n \in \mathbb{N}$. Aus der Dynkin-Formel(Satz $\ref{Dynkin-Formel}$) angewendet auf $T = S_{A} \wedge n$ und $f = h \wedge n$ folgt zusammen mit dem Lemma von Fatou
\begin{equation*}
h(y) = \liminf_{m \to \infty} h(y) \wedge m \geq \liminf_{m \to \infty}
\liminf_{n \to \infty} \mathbb{E}_{y}[h(X_{S_{A} \wedge n}) \wedge m] 
\end{equation*}
\begin{equation*}
\geq \mathbb{E}_{y}[h(X_{S_{A}})\mathbbm{1}_{S_{A} < \infty}] \geq \inf_{z \in A} h(z) \mathbb{P}_{y}[S_{A} < \infty]
\end{equation*}
Also,
\begin{equation*}
\mathbb{P}_{y}[S_{A}< \infty] \leq \dfrac{h(y)}{\inf_{z \in A} h(z)} < 1
\end{equation*}
Zusammen mit Lemma $\ref{Px[SA < unendlich] < 1, so y transient}$ a) folgt somit, dass jeder Zustand in E transient ist.
\item[b)] \dashuline{zu zeigen}: Für jedes $c \geq 0$ gilt $\mathbb{P}_{x}[S_{\lbrace h > c \rbrace} < \infty] = 1 \quad \forall x \in \lbrace h \leq c \rbrace$
\\
\\
Aus der Irreduziblität folgt zunächst einmal, dass für jedes $c \geq 0$
\begin{equation*}
\mathbb{P}_{x}[S_{\lbrace h > c \rbrace} = \infty] < 1 \qquad \forall x \in \lbrace h \leq c \rbrace
\end{equation*}
Folglich existiert zu jedem $x \in \lbrace h \leq c \rbrace$ ein $N_{x} \in \mathbb{N}$ und $\epsilon_{x} > 0$ so, dass
\begin{equation*}
\mathbb{P}_{x}[S_{\lbrace h > c \rbrace} > n] \leq 1 - \epsilon_{x} \qquad n \geq N_{x}
\end{equation*}
Setze $N := \max {\lbrace N_{x} \: : \: x \in {\lbrace h \leq c \rbrace}  \rbrace}$ und $\epsilon := \min \lbrace \epsilon_{x} \: : \: x \in \lbrace h \leq c \rbrace \rbrace$. Da nach Voraussetzungen die Menge $\lbrace h \leq c \rbrace$ endlich ist für jedes $c \geq 0$, folgt $N < \infty$, $\epsilon > 0$ und
\begin{equation*}
\mathbb{P}_{x}[S_{\lbrace h > c \rbrace} > n] \leq 1 - \epsilon \quad \forall n \geq N \quad \mathrm{und} \quad x \in {\lbrace h \leq x \rbrace}
\end{equation*}
Weiterhin gilt für jedes $k \in \mathbb{N}$
\begin{equation*}
\mathbb{P}_{x}[S_{\lbrace h > c \rbrace}> kN]
\end{equation*}
\begin{equation*}
= \sum_{y \in {\lbrace h \leq c \rbrace}} \mathbb{P}_{x}[S_{\lbrace h > c \rbrace}> kN \: | \: S_{\lbrace h > c \rbrace} > (k-1)N, X_{(k-1)N}=y] \mathbb{P}_{x}[S_{\lbrace h > c \rbrace} > (k-1)N, X_{(k-1)N}=y]
\end{equation*}
\begin{equation*}
\stackrel{\mathrm{Satz} \: \ref{vorangegangene und zukünftige Ereignisse}}{=} \sum_{y \in {\lbrace h \leq c \rbrace}} \underbrace{\mathbb{P}_{y}[S_{\lbrace h > c \rbrace}> N]}_{\leq 1 - \epsilon} \mathbb{P}_{x}[S_{\lbrace h > c \rbrace} > (k-1)N, X_{(k-1)N}=y]
\end{equation*}
\begin{equation*}
\leq (1-\epsilon) \mathbb{P}_{x}[S_{\lbrace h > c \rbrace} > (k-1)N]
\end{equation*}
Folglich ergibt sich induktiv, dass $\mathbb{P}_{x}[S_{\lbrace h > c \rbrace} > (k-1)N] \leq (1-\epsilon)^{k}$ für alle $k \in \mathbb{N}$ und $x \in \lbrace h \leq c \rbrace$.
\\
Somit erhält man 
\begin{equation*}
\mathbb{P}_{x}[S_{\lbrace h > c \rbrace} = \infty] = \limsup_{k \to \infty} \mathbb{P}_{x}[S_{\lbrace h > c \rbrace} = kN] \leq \limsup_{k \to \infty} (1-\epsilon)^{k} = 0 \quad \forall x \in {\lbrace h > c \rbrace}
\end{equation*}
\dashuline{zu zeigen}: $\mathbb{P}_{x}[S_{\lbrace h > c \rbrace} < \infty] = 1 \quad \forall x \in A$
\\
\\
Für jedes $c \geq 0$ gilt für die Stoppzeit $S_{A} \wedge n \wedge S_{{\lbrace h > c \rbrace}}$, dass $\mathbb{E}_{x}[S_{A} \wedge n \wedge S_{{\lbrace h > c \rbrace}}] \leq n < \infty$ für alle $n \in \mathbb{N}$ und $x \in E$. Mit der Dynkin-Formel (Satz $\ref{Dynkin-Formel}$) angewendet auf $T=S_{A} \wedge n \wedge S_{{\lbrace h > c \rbrace}}$ und $f = h \wedge m$ folgt zusammen mit dem Lemma von Fatou
\begin{equation*}
h(x) = \liminf_{m \to \infty} h(x) \wedge m
\end{equation*} 
\begin{equation*}
\geq \liminf_{m \to \infty} \liminf_{n \to \infty} \mathbb{E}_{x}[h(X_{S_{A} \wedge n \wedge S_{\lbrace h > c \rbrace}} ) \wedge m] \geq \mathbb{E}_{x}[h(X_{S_{A} \wedge n \wedge S_{\lbrace h > c \rbrace}} )]
\end{equation*}
Daraus folgt
\begin{equation*}
h(x) \geq \mathbb{E}_{x}[h(X_{S_{{\lbrace h > c \rbrace}}}) \mathbbm{1}_{S_{{\lbrace h > c \rbrace} < \infty}}\mathbbm{1}_{S_{{\lbrace A \rbrace} = \infty}}] \geq c \cdot \mathbb{P}_{x}[S_{{\lbrace h > c \rbrace}} < \infty, S_{{\lbrace h > c \rbrace}} = \infty]
\end{equation*}
\begin{equation*}
= c \cdot \mathbb{P}_{x}[S_{A} = \infty]
\end{equation*}
Da dies für jedes $c \geq 0$ gilt, folgt schließlich
\begin{equation*}
\mathbb{P}_{x}[S_{A} = \infty] \leq \limsup_{c \to \infty} \dfrac{h(x)}{c} = 0 \quad \Leftrightarrow \quad \mathbb{P}_{x}[S_{A} < \infty] = 1 \qquad \forall x \in E
\end{equation*}
Folglich ist nach Lemma $\ref{Px[SA < unendlich] < 1, so y transient}$ b) jeder Zustand $y \in E$ rekurrent.
\end{itemize}

\textbf{Beispiel 2.13}[Einfache Irrfahrt auf $\mathbb{Z}$] Sei $(X_{n})_{n \in \mathbb{N}_{0}}$ eine Markovkette auf $E = \mathbb{Z}$ mit folgendem Übergangsgraphen
\begin{figure}[H].
\centering
\includegraphics[scale=0.5]{einfache Irrfahrt auf Z}
\caption{Einfache Irrfahrt auf $\mathbb{Z}$}
\end{figure}
\noindent
Betrachte zunächst den Fall $p \neq \dfrac{1}{2}$. Dann gilt für $h(x) = {\left(\dfrac{1-p}{p} \right)}^{x}$, $x \in E$
\begin{equation*}
(Lh)(x) = p(h((x+1)-h(x)) + (1-p)(h(x-1)-h(x)) = h(x)(1-2p+(2p-1)) = 0
\end{equation*}
für alle $x \in E$. Wähle nun $A = \lbrace 0 \rbrace$ und 
\begin{equation*}
y =
\begin{cases}
\: \: \:1 & , p > \dfrac{1}{2}\\
& \\
-1 & , p < \dfrac{1}{2}
\end{cases}
\end{equation*}
Dann gilt 
\begin{equation*}
(Lh)(x) = 0 \quad \forall \: x \in A^{C} \quad und \quad h(y) < h(0) = 1
\end{equation*}
Somit ist die Bedingung (LT) erfüllt. Da zudem $(X_{n})_{n \in \mathbb{N}_{0}}$ irreduzibel ist, folgt aus Satz $\ref{Folgerung Dynkin Formel, (LR), (LT)}$ a), dass jeder Zustand transient ist.
\\
Im Falle $p = \dfrac{1}{2}$ betrachte die Funktion $h(x) = \vert x \vert$. Dann gilt
\begin{equation*}
(Lh)(x) = \dfrac{1}{2} (\vert x +1 \vert - \vert x \vert) + \dfrac{1}{2}(\vert x -1 \vert - \vert x \vert) =
\begin{cases}
0 & , x \neq 0\\

1 & , x = 0
\end{cases}
\end{equation*}
Wähle nun $A = \lbrace 0 \rbrace$. Dann gilt
\begin{equation*}
(Lh)(x) = 0 \quad \forall \: x \in A^{C} \quad und \quad \vert \lbrace h \leq c\rbrace \vert < \infty \quad \forall \: c \geq 0.
\end{equation*}
Somit ist die Bedingung (LR) erfüllt. Da $(X_{n})_{n \in \mathbb{N}_{0}}$ zudem irreduzibel ist, ist folglich nach Satz $\ref{Folgerung Dynkin Formel, (LR), (LT)}$ b) jeder Zustand rekurrent.\mbox{}
\begin{itemize}
\item[a)] Offensichtlich gilt für die Stoppzeit $S_{A} \wedge n$, dass $\mathbb{E}_{x}[S_{A} \wedge n] \leq n$ für alle $x \in E$ und $n \in \mathbb{N}$. Aus der Dynkin-Formel(Satz $\ref{Dynkin-Formel}$) angewendet auf $T = S_{A} \wedge n$ und $f = h \wedge n$ folgt zusammen mit dem Lemma von Fatou
\begin{equation*}
h(y) = \liminf_{m \to \infty} h(y) \wedge m \geq \liminf_{m \to \infty}
\liminf_{n \to \infty} \mathbb{E}_{y}[h(X_{S_{A} \wedge n}) \wedge m] 
\end{equation*}
\begin{equation*}
\geq \mathbb{E}_{y}[h(X_{S_{A}})\mathbbm{1}_{S_{A} < \infty}] \geq \inf_{z \in A} h(z) \mathbb{P}_{y}[S_{A} < \infty]
\end{equation*}
Also,
\begin{equation*}
\mathbb{P}_{y}[S_{A}< \infty] \leq \dfrac{h(y)}{\inf_{z \in A} h(z)} < 1
\end{equation*}
Zusammen mit Lemma $\ref{Px[SA < unendlich] < 1, so y transient}$ a) folgt somit, dass jeder Zustand in E transient ist.
\item[b)] \dashuline{zu zeigen}: Für jedes $c \geq 0$ gilt $\mathbb{P}_{x}[S_{\lbrace h > c \rbrace} < \infty] = 1 \quad \forall x \in \lbrace h \leq c \rbrace$
\\
\\
Aus der Irreduziblität folgt zunächst einmal, dass für jedes $c \geq 0$
\begin{equation*}
\mathbb{P}_{x}[S_{\lbrace h > c \rbrace} = \infty] < 1 \qquad \forall x \in \lbrace h \leq c \rbrace
\end{equation*}
Folglich existiert zu jedem $x \in \lbrace h \leq c \rbrace$ ein $N_{x} \in \mathbb{N}$ und $\epsilon_{x} > 0$ so, dass
\begin{equation*}
\mathbb{P}_{x}[S_{\lbrace h > c \rbrace} > n] \leq 1 - \epsilon_{x} \qquad n \geq N_{x}
\end{equation*}
Setze $N := \max {\lbrace N_{x} \: : \: x \in {\lbrace h \leq c \rbrace}  \rbrace}$ und $\epsilon := \min \lbrace \epsilon_{x} \: : \: x \in \lbrace h \leq c \rbrace \rbrace$. Da nach Voraussetzungen die Menge $\lbrace h \leq c \rbrace$ endlich ist für jedes $c \geq 0$, folgt $N < \infty$, $\epsilon > 0$ und
\begin{equation*}
\mathbb{P}_{x}[S_{\lbrace h > c \rbrace} > n] \leq 1 - \epsilon \quad \forall n \geq N \quad \mathrm{und} \quad x \in {\lbrace h \leq x \rbrace}
\end{equation*}
Weiterhin gilt für jedes $k \in \mathbb{N}$
\begin{equation*}
\mathbb{P}_{x}[S_{\lbrace h > c \rbrace}> kN]
\end{equation*}
\begin{equation*}
= \sum_{y \in {\lbrace h \leq c \rbrace}} \mathbb{P}_{x}[S_{\lbrace h > c \rbrace}> kN \: | \: S_{\lbrace h > c \rbrace} > (k-1)N, X_{(k-1)N}=y] \mathbb{P}_{x}[S_{\lbrace h > c \rbrace} > (k-1)N, X_{(k-1)N}=y]
\end{equation*}
\begin{equation*}
\stackrel{\mathrm{Satz} \: \ref{vorangegangene und zukünftige Ereignisse}}{=} \sum_{y \in {\lbrace h \leq c \rbrace}} \underbrace{\mathbb{P}_{y}[S_{\lbrace h > c \rbrace}> N]}_{\leq 1 - \epsilon} \mathbb{P}_{x}[S_{\lbrace h > c \rbrace} > (k-1)N, X_{(k-1)N}=y]
\end{equation*}
\begin{equation*}
\leq (1-\epsilon) \mathbb{P}_{x}[S_{\lbrace h > c \rbrace} > (k-1)N]
\end{equation*}
Folglich ergibt sich induktiv, dass $\mathbb{P}_{x}[S_{\lbrace h > c \rbrace} > (k-1)N] \leq (1-\epsilon)^{k}$ für alle $k \in \mathbb{N}$ und $x \in \lbrace h \leq c \rbrace$.
\\
Somit erhält man 
\begin{equation*}
\mathbb{P}_{x}[S_{\lbrace h > c \rbrace} = \infty] = \limsup_{k \to \infty} \mathbb{P}_{x}[S_{\lbrace h > c \rbrace} = kN] \leq \limsup_{k \to \infty} (1-\epsilon)^{k} = 0 \quad \forall x \in {\lbrace h > c \rbrace}
\end{equation*}
\dashuline{zu zeigen}: $\mathbb{P}_{x}[S_{\lbrace h > c \rbrace} < \infty] = 1 \quad \forall x \in A$
\\
\\
Für jedes $c \geq 0$ gilt für die Stoppzeit $S_{A} \wedge n \wedge S_{{\lbrace h > c \rbrace}}$, dass $\mathbb{E}_{x}[S_{A} \wedge n \wedge S_{{\lbrace h > c \rbrace}}] \leq n < \infty$ für alle $n \in \mathbb{N}$ und $x \in E$. Mit der Dynkin-Formel (Satz $\ref{Dynkin-Formel}$) angewendet auf $T=S_{A} \wedge n \wedge S_{{\lbrace h > c \rbrace}}$ und $f = h \wedge m$ folgt zusammen mit dem Lemma von Fatou
\begin{equation*}
h(x) = \liminf_{m \to \infty} h(x) \wedge m
\end{equation*} 
\begin{equation*}
\geq \liminf_{m \to \infty} \liminf_{n \to \infty} \mathbb{E}_{x}[h(X_{S_{A} \wedge n \wedge S_{\lbrace h > c \rbrace}} ) \wedge m] \geq \mathbb{E}_{x}[h(X_{S_{A} \wedge n \wedge S_{\lbrace h > c \rbrace}} )]
\end{equation*}
Daraus folgt
\begin{equation*}
h(x) \geq \mathbb{E}_{x}[h(X_{S_{{\lbrace h > c \rbrace}}}) \mathbbm{1}_{S_{{\lbrace h > c \rbrace} < \infty}}\mathbbm{1}_{S_{{\lbrace A \rbrace} = \infty}}] \geq c \cdot \mathbb{P}_{x}[S_{{\lbrace h > c \rbrace}} < \infty, S_{{\lbrace h > c \rbrace}} = \infty]
\end{equation*}
\begin{equation*}
= c \cdot \mathbb{P}_{x}[S_{A} = \infty]
\end{equation*}
Da dies für jedes $c \geq 0$ gilt, folgt schließlich
\begin{equation*}
\mathbb{P}_{x}[S_{A} = \infty] \leq \limsup_{c \to \infty} \dfrac{h(x)}{c} = 0 \quad \Leftrightarrow \quad \mathbb{P}_{x}[S_{A} < \infty] = 1 \qquad \forall x \in E
\end{equation*}
Folglich ist nach Lemma $\ref{Px[SA < unendlich] < 1, so y transient}$ b) jeder Zustand $y \in E$ rekurrent.
\end{itemize}

\textbf{Beispiel 2.13}[Einfache Irrfahrt auf $\mathbb{Z}$] Sei $(X_{n})_{n \in \mathbb{N}_{0}}$ eine Markovkette auf $E = \mathbb{Z}$ mit folgendem Übergangsgraphen
\begin{figure}[H].
\centering
\includegraphics[scale=0.5]{einfache Irrfahrt auf Z}
\caption{Einfache Irrfahrt auf $\mathbb{Z}$}
\end{figure}
\noindent
Betrachte zunächst den Fall $p \neq \dfrac{1}{2}$. Dann gilt für $h(x) = {\left(\dfrac{1-p}{p} \right)}^{x}$, $x \in E$
\begin{equation*}
(Lh)(x) = p(h((x+1)-h(x)) + (1-p)(h(x-1)-h(x)) = h(x)(1-2p+(2p-1)) = 0
\end{equation*}
für alle $x \in E$. Wähle nun $A = \lbrace 0 \rbrace$ und 
\begin{equation*}
y =
\begin{cases}
\: \: \:1 & , p > \dfrac{1}{2}\\
& \\
-1 & , p < \dfrac{1}{2}
\end{cases}
\end{equation*}
Dann gilt 
\begin{equation*}
(Lh)(x) = 0 \quad \forall \: x \in A^{C} \quad und \quad h(y) < h(0) = 1
\end{equation*}
Somit ist die Bedingung (LT) erfüllt. Da zudem $(X_{n})_{n \in \mathbb{N}_{0}}$ irreduzibel ist, folgt aus Satz $\ref{Folgerung Dynkin Formel, (LR), (LT)}$ a), dass jeder Zustand transient ist.
\\
Im Falle $p = \dfrac{1}{2}$ betrachte die Funktion $h(x) = \vert x \vert$. Dann gilt
\begin{equation*}
(Lh)(x) = \dfrac{1}{2} (\vert x +1 \vert - \vert x \vert) + \dfrac{1}{2}(\vert x -1 \vert - \vert x \vert) =
\begin{cases}
0 & , x \neq 0\\

1 & , x = 0
\end{cases}
\end{equation*}
Wähle nun $A = \lbrace 0 \rbrace$. Dann gilt
\begin{equation*}
(Lh)(x) = 0 \quad \forall \: x \in A^{C} \quad und \quad \vert \lbrace h \leq c\rbrace \vert < \infty \quad \forall \: c \geq 0.
\end{equation*}
Somit ist die Bedingung (LR) erfüllt. Da $(X_{n})_{n \in \mathbb{N}_{0}}$ zudem irreduzibel ist, ist folglich nach Satz $\ref{Folgerung Dynkin Formel, (LR), (LT)}$ b) jeder Zustand rekurrent.

\textbf{Beispiel 2.13}[Einfache, symmetrische Irrfahrt auf $\mathbb{Z}^{d}$, $d \geq 3$]
Sei $(X_{n})_{n \in \mathbb{N}_{0}}$ eine Markovkette auf $E = \mathbb{Z}^{d}$, $d \geq 3$ mit folgender Übergangswahrscheinlichkeiten
\begin{equation*}
p(x,y)=
\begin{cases}
\dfrac{1}{2d} & ,  \vert \vert x - y \vert \vert = 1\\
0 & , sonst
\end{cases}
\end{equation*}
Betrachte nun die Funktion $h(0)=1$ und $h(x)= {\vert \vert x \vert \vert}_{2}^{-\alpha}, \: x \neq 0$. Dann gilt für jedes $x \in \mathbb{Z}^{d}$ mit ${\vert \vert x \vert \vert}_{2} > 1$ und $e \in \mathbb{Z}^{d}$ mit ${\vert \vert e \vert \vert}_{2} = 1$
\begin{equation*}
h(x+e) - h(x) = h(x)\left(\left(\dfrac{{\vert \vert x + e \vert \vert}_{2}^{2}}{{\vert \vert x \vert \vert}_{2}^{2}}\right)^{-\alpha} - 1 \right) = h(x)\left(\left( 1 + \dfrac{{2\langle x,e \rangle + 1}}{{\vert \vert x \vert \vert}_{2}^{2}}\right)^{-\alpha} - 1 \right)
\end{equation*}
\begin{equation*}
 = h(x)\left(1 - \alpha \dfrac{{2\langle x,e \rangle + 1}}{{\vert \vert x \vert \vert}_{2}^{2} } + 2\alpha(\alpha + 1)\dfrac{{{\langle x,e \rangle}^{2}}}{{\vert \vert x \vert \vert}_{2}^{4} } + \mathcal{O}({\vert \vert x \vert \vert}_{2}^{-3}) - 1 \right)
\end{equation*}
wobei die Taylorentwicklung der Funktion $f(z)=(1+z)^{-\alpha}= 1 - \alpha z + \dfrac{1}{2} \alpha(\alpha +1)z^{2} + \mathcal{O}({\vert z \vert}^{3})$ benutzt wurde. Da zudem gilt
\begin{equation*}
\sum_{{\vert \vert e \vert \vert}_{2} = 1} 1 = 2d, \quad \sum_{{\vert \vert e \vert \vert}_{2} = 1} \langle x,e \rangle = 0 \quad und \quad \sum_{{\vert \vert e \vert \vert}_{2} = 1} {\langle x,e \rangle}^{2} = 2 {\vert \vert x \vert \vert}_{2}^{2}
\end{equation*}
folgt
\begin{equation*}
\sum_{{\vert \vert e \vert \vert}_{2} = 1} \dfrac{1}{2d} \left(h(x+e) - h(x)\right) = \dfrac{1}{2d}h(x)(-2 \alpha d {\vert \vert x \vert \vert}_{2}^{-2} + 4 \alpha (\alpha + 1){\vert \vert x \vert \vert}_{2}^{-2} + \mathcal{O}({\vert \vert x \vert \vert}_{2}^{-3}) )
\end{equation*}
\begin{equation*}
= \dfrac{\alpha}{d} {\vert \vert x \vert \vert}_{2}^{-2 \alpha - 2} (2(\alpha + 1) - d + \mathcal{O}({\vert \vert x \vert \vert}_{2}^{-1}))
\end{equation*}
Daraus folgt für $d \geq 3$ $\alpha \in \left(0,\dfrac{d-2}{2}\right)$ und $A := \lbrace x \: : \: {\vert \vert x \vert \vert}_{2} \leq r \rbrace$ für ein hinreichend großes $r>0$, dass
\begin{equation*}
(Lh)(x) \leq 0 \quad \forall \: x \in A^{C} \quad und \quad h(y) < \inf_{z \in A} h(z) \quad für \: ein \: y \: mit \: {\vert \vert y \vert \vert}_{2} > 2r.
\end{equation*}
Da $(X_{n})_{n \in \mathbb{N}_{0}}$ zudem irreduzibel ist, folgt aus Satz $\ref{Folgerung Dynkin Formel, (LR), (LT)}$ a), dass die einfache, symmetrische Irrfahrt auf $E = \mathbb{Z}^{d}$ für jedes $d \geq 3$ transient ist.