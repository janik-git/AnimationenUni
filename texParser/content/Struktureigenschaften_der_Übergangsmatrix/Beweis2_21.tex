Für $x \in E$ sei $D(x) := \lbrace n \in \mathbb{N}_{0} \: : \: p_{n}(x,x)>0 \rbrace$. Falls $D(x) \neq \lbrace 0 \rbrace$, so ergibt sich für alle $n_{1},n_{2} \in D(x)$ aus der Chapman-Kolmogorov-Gleichung (Satz $\ref{Chapman-Kolmogorov Gleichung}$)
\begin{equation*}
p_{n_{1} + n_{2}}(x,x) = \sum_{z \in E} p_{n_{1}}(x,z)p_{n_{2}}(z,x) \geq p_{n_{1}}(x,x)p_{n_{2}}(x,x) > 0 
\end{equation*}
Also, $n_{1} + n_{2} \in D(x)$. Folglich ist $D(x)$ abgeschlossen unter der Addition.
\begin{itemize}
\item[a)] $d(x) < \infty$ $\Rightarrow$ $D(x) \neq \lbrace 0 \rbrace$ $\stackrel{\mathrm{Satz} \: \ref{Teiler als Linearkombi} \: b)}{\Rightarrow}$ $\: \exists \: N(x) \in \mathbb{N} \: : \: p_{n \cdot d(x)}(x,x) > 0, \: \forall n \geq N(x)$
\item[b)] $"\Rightarrow"$ Folgt direkt aus a)
\\
$"\Leftarrow"$ Sei also nun $p_{n}(x,x)>0$ für alle $n \geq N(x) \in \mathbb{N}$. Dann enthält $D(x)$ unendlich viele Primzahlen. Folglich ist $d(x)=ggT(D(x))=1$
\end{itemize}