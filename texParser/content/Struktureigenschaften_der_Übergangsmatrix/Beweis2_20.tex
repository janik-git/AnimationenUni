a) Sei \^{D} die kleinste Teilmenge von $\mathbb{Z}$ mit der Eigenschaft, dass
\begin{equation*}
D \subseteq \hat{D} \quad \mathrm{und} \quad \forall d_{1},d_{2} \in \hat{D} \quad \Rightarrow \quad d_{1} \pm d_{2} \in \hat{D}
\end{equation*}
Betrachte nun die Menge
\begin{equation*}
\mathcal{D} := \lbrace \hat{d} \in \hat{D} \: : \: \exists k \in \mathbb{N}, \: a_{1},...,a_{k} \in \mathbb{Z}, \: d_{1},...,d_{k} \in D \: \: s.d. \: \: \hat{d} = \sum_{i=1}^{k} a_{i}d_{i} \rbrace
\end{equation*}
\dashuline{zu zeigen}: $\hat{D} = \mathcal{D}$
\\
\\
Zunächst einmal gilt $D \subseteq \mathcal{D}$. Betrachte nun $x,y \in \mathcal{D}$. Dann gilt $x \pm y \in \mathcal{D}$. Da aber $\hat{D}$ die kleinste Teilmenge ist, die D enthält und abgeschlossen bzgl. Addition/Subtraktion ist, folgt $\hat{D} \subseteq \mathcal{D}$. Also $\hat{D} = \mathcal{D}$
\\
\\
\dashuline{zu zeigen}: $ggT(D) = ggT(\hat{D})$
\\
\\
Da $ggT(D) \: | \: \sum_{i=1}^{k} a_{i}d_{i}$ für alle $k \in \mathbb{N}$, $a_{1},...,a_{k} \in \mathbb{Z}$, $d_{1},...,d_{k} \in D$ folgt $ggT(D) \: | \: \hat{D}$. Also
\begin{equation*}
ggT(D) \leq ggT(\hat{D})
\end{equation*}
Adererseits ist $D \subseteq \hat{D}$, weshalb $ggT(\hat{D}) \: | \: D$. Also,
\begin{equation*}
ggT(\hat{D}) \leq ggT(D)
\end{equation*}
und somit gilt $ggT(D) = ggT(\hat{D})$.
\\
\\
\dashuline{zu zeigen}: $ggT(\hat{D}) \in \hat{D}$
\\
\\
Setze $m := \min \lbrace x \in \mathbb{N} \: : \: x \in \hat{D} \rbrace$. Aufgrund von Bemerkung $\ref{Bemerkung zu Teilern}$ gilt $ggT(\hat{D}) \leq m$. Durch die Anwendung des euklidischen Algorithmus ergibt sich für jedes $\hat{d} \in \hat{D}$, dass
\begin{equation*}
\hat{d} = a \cdot m + r \qquad \mathrm{für} \: a \in \mathbb{Z} \: \mathrm{und} \: r \in \lbrace 0,...,m-1 \rbrace
\end{equation*}
Also,
\begin{equation*}
r = \underbrace{\hat{d}}_{\in \hat{D}} - \underbrace{a \cdot m}_{\in \hat{D}} \in \hat{D}
\end{equation*}
Angenommen $r \neq 0$, so ist $r<m$ $\lightning$. Also ist $m \: | \: \hat{D}$ und folglich $m \leq ggT(\hat{D})$. Somit gilt $m=ggT(\hat{D})$ und $ggT(\hat{D}) \in \hat{D}$
\\
\\
b) Sei zusätzlich angenommen, dass $D \subseteq \mathbb{N}_{0}$ und $d_{1}, d_{2} \in D$ $\Rightarrow$ $d_{1} + d_{2} \in D$.\\
\\
\\
\dashuline{zu zeigen}: $\exists \: N \in \mathbb{N}:$ $\lbrace
 n \cdot ggT(D) \: : \: n \geq N \rbrace$ $\subseteq$ D
\\
\\
Da D abgeschlossen unter Addition ist, folgt $\hat{D} = \lbrace d_{2} - d_{1} \: : \: d_{1},d_{2} \in D \cup \lbrace 0 \rbrace \rbrace$. Da
\begin{equation*}
ggT(D) = ggT(\hat{D}) = \min \lbrace x \in \mathbb{N} \: : \: x \in \hat{D} \rbrace \quad (\mathrm{nach \: Beweisteil \: a}))
\end{equation*}
gibt es somit $d_{1} \in D \cup \lbrace 0 \rbrace$ und $d_{2} \in D$, $d_{2} > d_{1}$ so, dass
\begin{equation*}
ggT(D) = d_{2} - d_{1}
\end{equation*}
Falls $d_{1} = 0$, so gilt
\begin{equation*}
n \cdot ggT(D) = n \cdot d_{2} \in D \quad \forall n \in \mathbb{N} \quad \Rightarrow \quad N=1
\end{equation*}
Falls $d_{1} \neq 0$, so wähle ein $a \in \mathbb{N}$ mit $d_{1} = a \cdot ggT(D)$. Dann gilt für alle $m,r \in \mathbb{N}_{0}$ mit $0 \leq r < m$
\begin{equation*}
(a^{2} + m \cdot a +r)\cdot ggT(D) = (a+m)\cdot a \cdot ggT(D) + r \cdot ggT(D) = (a+m) \cdot d_{1} + r \cdot d_{2} \in D
\end{equation*}
Wähle somit $N = a^{2}$. Dann gilt $\lbrace n \cdot ggT(D) \: : \: n \geq N \rbrace \subseteq D.$ 