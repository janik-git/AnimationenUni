a) \dashuline{zu zeigen}: $x \leftrightarrow y \quad \Rightarrow \quad d(x) = d(y)$
\\
\\
Bezeichne mit $D(x) := \lbrace n \in \mathbb{N}_{0} \: : \: p_{n}(x,x)>0 \rbrace$, $x \in E$. Seien nun $x,y \in E$ mit x $\leftrightarrow$ y. Wähle $m,n \in \mathbb{N}_{0}$ so, dass $p_{m}(x,y)>0$ und $p_{n}(y,x)>0$. Dann gilt für jedes $k \in D(y)$ aufgrund der Chapman-Kolmogorov-Gleichung
\begin{equation*}
p_{m+k+n}(x,x) \geq p_{m}(x,y) \cdot p_{k}(y,y) \cdot p_{n}(y,x) > 0
\end{equation*}
Folglich ist $m+k+n \in D(x)$. Ist nun d ein Teiler von D(x), so gilt
\begin{equation*}
 d \: | \: \lbrace m+k+n \: : \: k \in D(y) \rbrace
\end{equation*} 
Da aber $m+n \in D(x)$ und somit $d \: | \: (m+n)$, folgt $d \: | \: D(y)$. Also, 
\begin{equation*}
d(x) = ggT(D(x)) \leq ggT(D(y)) = d(y)
\end{equation*}
Durch Vertauschen der Rollen von x und y folgt analog $d(y) \leq d(x)$. Also $d(x) = d(y)$.
\\
\\
b) $"\Rightarrow"$ Sei x transient. Da x $\leftrightarrow$ y, existieren $k,l \in \mathbb{N}$ so, dass  $p_{k}(x,y)>0$ und $p_{l}(y,x)>0$. Dann folgt aus der Chapman-Kolmogorov-Gleichung
\begin{equation*}
p_{k+l+n}(x,x) \geq p_{k}(x,y) \cdot p_{n}(y,y) \cdot p_{l}(y,x) > 0 \quad \forall n \in \mathbb{N}.
\end{equation*}
Daraus folgt aus Satz $\ref{alternative Chrakterisierung von rekurrent/transient}$ 
\begin{equation*}
\infty > \sum_{n=1}^{\infty} p_{k+l+n}(x,x) \geq \underbrace{p_{k}(x,y)}_{>0} \cdot \underbrace{p_{l}(y,x)}_{>0} \sum_{n=1}^{\infty} p_{n}(y,y) \quad \Rightarrow \quad \sum_{n=1}^{\infty} p_{n}(y,y) < \infty
\end{equation*}
Also ist y transient.
\\
$"\Leftarrow"$ Analog.
\\
\\
c) \: $"\Rightarrow"$ Sei x nullrekurrent. Da $x \leftrightarrow y$ existieren somit $k,l \in \mathbb{N}$ mit $p_{k}(x,y) > 0$ und $p_{l}(y,x) > 0$. Wiederum folgt aus der Chapman-Kolmogorov-Gleichung
\begin{equation*}
p_{k+l+n}(x,x) \geq p_{k}(x,y) \cdot p_{n}(y,y) \cdot p_{l}(y,x) \quad \forall n \in \mathbb{N}
\end{equation*}
Dann folgt aus Satz $\ref{nullrekurrent und limes}$
\begin{equation*}
0 = \lim_{n \to \infty}p_{k+l+n}(x,x) \geq \limsup_{n \to \infty} \underbrace{p_{k}(x,y)}_{>0} \cdot \underbrace{p_{l}(y,x)}_{>0} \cdot p_{n}(y,y) \quad \Rightarrow \quad \lim_{n \to \infty} p_{n}(y,y) = 0
\end{equation*}
Folglich ist y nach Satz $\ref{nullrekurrent und limes}$ nullrekurrent.
\\
$"\Leftarrow"$ Analog.