Sei $(X_{n})_{n \in \mathbb{N}_{0}}$ eine $(\nu,P)$-Markovkette mit Zustandsraum E. Angenommen der Zustand $x \in E$ ist periodisch $(d(x) \geq 2)$. Betrachte man die Markovkette $(X'_{n})_{n \in \mathbb{N}_{0}}$ mit Startverteilung $\nu$ und Übergangsmatrix $P' = \epsilon I + (1-\epsilon)P, \epsilon \in (0,1)$, wobei I die Einheitsmatrix auf E ist. Dann gilt $d(x)=1$ für alle $x \in E$, da
\begin{equation*}
\lbrace 1 \rbrace \in \lbrace n \in \mathbb{N}_{0} \: : \: p'_{n}(x,x)>0 \rbrace
\end{equation*}
Die Markovkette  $(X'_{n})_{n \in \mathbb{N}_{0}}$ nennt man auch träge Markovkette.