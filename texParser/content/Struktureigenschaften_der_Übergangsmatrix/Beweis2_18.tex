Sei $y \in E$ ein transienter Zustand. Dann folgt aus Satz $\ref{alternative Chrakterisierung von rekurrent/transient}$
\begin{equation*}
\sum_{n=1}^{\infty} p_{n}(y,y) < \infty \: \Rightarrow \: \lim_{n \to \infty} p_{n}(y,y) = 0
\end{equation*}
Sei also nun $x \in E$, $x \neq y$. Dann gilt
\begin{equation*}
p_{n}(x,y) = \mathbb{P}_{x}[X_{n} = y] = \sum_{k=1}^{n} \mathbb{P}_{x}[X_{n} = y, S_{\lbrace y \rbrace} = k]
\end{equation*}
\begin{equation*}
= \sum_{k=1}^{n} \mathbb{P}_{x}[S_{\lbrace y \rbrace} = k] \cdot \mathbb{P}_{x}[X_{n} = y \: | \: X_{k} = y, S_{\lbrace y \rbrace} = k] 
\end{equation*}
Somit folgt aus Satz $\ref{vorangegangene und zukünftige Ereignisse}$
\begin{equation*}
p_{n}(x,y) = \sum_{k=1}^{n} \mathbb{P}_{x}[S_{\lbrace y \rbrace} = k] \cdot \mathbb{P}_{y}[X_{n-k} = y] = \sum_{k=1}^{n} \mathbb{P}_{x}[S_{\lbrace y \rbrace} = k] \cdot p_{n-k}(y,y)
\end{equation*}
Also,
\begin{equation*}
\sum_{n=1}^{\infty} p_{n}(x,y) = \sum_{k=1}^{\infty} \mathbb{P}_{x}[S_{\lbrace y \rbrace} = k] \sum_{n=k}^{\infty} p_{n-k}(y,y) = \mathbb{P}_{x}[S_{\lbrace y \rbrace} < \infty](1 + \underbrace{\sum_{n=1}^{\infty} p_{n}(y,y)}_{< \infty}) < \infty
\end{equation*}
Damit erhält man $\lim \limits_{n \to \infty} p_{n}(x,y) = 0$ für alle $x \neq y$.