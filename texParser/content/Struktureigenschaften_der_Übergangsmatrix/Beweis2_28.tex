Zunächst einmal gilt für jede beschränkte Funktion $f: E \to \mathbb{R}$ und jedes $n \in \mathbb{N}$
\begin{equation*}
(P^{n}f)(x) - f(x) = \sum_{k=0}^{n-1} (P^{k+1}f)(x) - (P^{k}f)(x) = \sum_{k=0}^{n-1} (P^{k} Lf)(x) \quad \forall x \in E.
\end{equation*}
Setze $g:= -Lh \geq 0$. Dann gilt nach Voraussetzungen, dass $g(y) > 0$. Daraus folgt
\begin{equation*}
h(y) \geq h(y) - (P^{n}h)(y) = \sum_{k=0}^{n-1} (P^{k}g)(y) = \sum_{k=0}^{n-1} \sum_{z \in E} p_{k}(y,z)g(z) \geq g(y) \sum_{k=0}^{n-1} p_{k}(y,y)
\end{equation*}
für jedes $n \in \mathbb{N}$. Also,
\begin{equation*}
\sum_{k=1}^{\infty} p_{k}(y,y) \leq \lim_{n \to \infty} \sum_{k=0}^{n-1} p_{k}(y,y) \leq \dfrac{h(y)}{g(y)} < \infty
\end{equation*}
Folglich ist der Zustand y nach Satz $\ref{alternative Chrakterisierung von rekurrent/transient}$ transient.Zunächst einmal gilt für jede beschränkte Funktion $f: E \to \mathbb{R}$ und jedes $n \in \mathbb{N}$
\begin{equation*}
(P^{n}f)(x) - f(x) = \sum_{k=0}^{n-1} (P^{k+1}f)(x) - (P^{k}f)(x) = \sum_{k=0}^{n-1} (P^{k} Lf)(x) \quad \forall x \in E.
\end{equation*}
Setze $g:= -Lh \geq 0$. Dann gilt nach Voraussetzungen, dass $g(y) > 0$. Daraus folgt
\begin{equation*}
h(y) \geq h(y) - (P^{n}h)(y) = \sum_{k=0}^{n-1} (P^{k}g)(y) = \sum_{k=0}^{n-1} \sum_{z \in E} p_{k}(y,z)g(z) \geq g(y) \sum_{k=0}^{n-1} p_{k}(y,y)
\end{equation*}
für jedes $n \in \mathbb{N}$. Also,
\begin{equation*}
\sum_{k=1}^{\infty} p_{k}(y,y) \leq \lim_{n \to \infty} \sum_{k=0}^{n-1} p_{k}(y,y) \leq \dfrac{h(y)}{g(y)} < \infty
\end{equation*}
Folglich ist der Zustand y nach Satz $\ref{alternative Chrakterisierung von rekurrent/transient}$ transient.
