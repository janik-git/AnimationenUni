Sei $(X_{n})_{n \in \mathbb{N}_{0}}$ eine Markovkette mit Zustandsraum $E = \mathbb{N}_{0}$, dessen Übergangsmatrix $P = (p(x,y))_{x,y \in E}$ durch folgenden Übergangsgraphen beschrieben wird
\begin{figure}[H].
\centering
\includegraphics[scale=0.5]{Beispiel_Kartenhausprozess}
\caption{Übergangsgraph des Kartenhausprozesses}
\end{figure}
\noindent
\textcolor{red}{Frage?} Unter welchen Bedingungen an $(p_{i})_{i \in \mathbb{N}_{0}}$ ist der Zustand $x=0$ rekurrent?
\\
\\
Da es für jedes $n \in \mathbb{N}$ genau einen Pfad gibt, der bei Start in $x=0$ nach genau n-Schritten wieder in 0 trifft (nämlich $(X_{0},X_{1},...,X_{n}) = (0,1,2,...,n-1,0)$) folgt
\begin{equation*}
\mathbb{P}_{0} [S_{\lbrace 0 \rbrace} = n] = (1-p_{0}) \cdot ... \cdot (1-p_{n-2}) \cdot p_{n-1} \: , \quad n \in \mathbb{N}
\end{equation*} 
Setze nun $u_{0} := 1$ und $u_{n} = (1-p_{0}) \cdot ... \cdot (1-p_{n-1})$, $\:$ $n \in \mathbb{N}$. Dann gilt
\begin{equation*}
\mathbb{P}_{0} [S_{\lbrace 0 \rbrace} = n] = u_{n-1} - u_{n}
\end{equation*}
Daraus folgt
\begin{equation*}
\mathbb{P}_{0} [S_{\lbrace 0 \rbrace} < \infty] = \lim_{N \to \infty} \sum_{n=1}^{N} \mathbb{P}_{0} [S_{\lbrace 0 \rbrace} = n] = \lim_{N \to \infty}(1-u_{N}) = 1 - \lim_{N \to \infty } \prod_{n=0}^{N-1}  (1 - p_{n})
\end{equation*}
\underline{Beh.}: Falls $p_{i} \in (0,1)$ für alle $i \in \mathbb{N}_{0}$ so gilt
\begin{equation*}
\lim_{N \to \infty} \prod_{n=0}^{N} (1-p_{n}) = 0 \quad \Leftrightarrow \quad \sum_{n=0}^{\infty} p_{n} = \infty
\end{equation*}
$"\Leftarrow"$ Da $e^{-x} \geq 1-x$ für alle $x \geq 0$, folgt
\begin{equation*}
0 \leq \lim_{N \to \infty} \prod_{n=0}^{N} (1-p_{n}) \leq \lim_{N \to \infty} exp(-\sum_{n=0}^{N}p_{n}) \stackrel{\sum_{n=0}^{\infty}p_{n} = \infty}{=} 0
\end{equation*}
$"\Rightarrow"$ $\dashuline{zu \: zeigen}:$ Für jedes $m \in \mathbb{N}_{0}$ gilt $\prod_{n=m}^{m+N}(1-p_{n}) \geq 1 - \sum_{n=m}^{m+N}p_{n}$, $\forall N \in \mathbb{N}_{0}$
\\
\\
\textbf{IA} $N=0$: $\checkmark$
\\
\\
\textbf{IS} $N \to N+1$: Es gilt nun aber
\begin{equation*}
\prod_{n=m}^{m+N+1} (1-p_{n}) \stackrel{\mathrm{IV}}{\geq} (1 - \sum_{n=m}^{m+N}p_{n})(1-p_{m+N+1}) = 1 - p_{m+N+1} - (\underbrace{1-p_{m+N+1}}_{\leq 1}) \cdot \sum_{n=m}^{m+N}p_{n} \geq 1 - \sum_{n=m}^{m+N+1} p_{n}
\end{equation*}
Angenommen $\sum_{n=0}^{\infty}p_{n} < \infty$. Dann existiert ein $m \in \mathbb{N}$ so, dass $0 < \sum_{n=m}^{\infty} p_{n} < 1$.
\\
Daraus folgt
\begin{equation*}
0 = \lim_{N \to \infty} \prod_{n=0}^{N} (1-p_{n}) = \prod_{n=0}^{m-1} (1-p_{n})  \lim_{N \to \infty} \prod_{n=m}^{N} (1-p_{n}) \geq \prod_{n=0}^{m-1} (1-p_{n}) \cdot (1 - \underbrace{\sum_{n=m}^{\infty} p_{n}}_{<1}) > 0 \: \: \lightning
\end{equation*}
Folglich ist $\sum_{n=0}^{\infty} p_{n} = \infty$.Sei $(X_{n})_{n \in \mathbb{N}_{0}}$ eine Markovkette mit Zustandsraum $E = \mathbb{N}_{0}$, dessen Übergangsmatrix $P = (p(x,y))_{x,y \in E}$ durch folgenden Übergangsgraphen beschrieben wird
\begin{figure}[H].
\centering
\includegraphics[scale=0.5]{Beispiel_Kartenhausprozess}
\caption{Übergangsgraph des Kartenhausprozesses}
\end{figure}
\noindent
\textcolor{red}{Frage?} Unter welchen Bedingungen an $(p_{i})_{i \in \mathbb{N}_{0}}$ ist der Zustand $x=0$ rekurrent?
\\
\\
Da es für jedes $n \in \mathbb{N}$ genau einen Pfad gibt, der bei Start in $x=0$ nach genau n-Schritten wieder in 0 trifft (nämlich $(X_{0},X_{1},...,X_{n}) = (0,1,2,...,n-1,0)$) folgt
\begin{equation*}
\mathbb{P}_{0} [S_{\lbrace 0 \rbrace} = n] = (1-p_{0}) \cdot ... \cdot (1-p_{n-2}) \cdot p_{n-1} \: , \quad n \in \mathbb{N}
\end{equation*} 
Setze nun $u_{0} := 1$ und $u_{n} = (1-p_{0}) \cdot ... \cdot (1-p_{n-1})$, $\:$ $n \in \mathbb{N}$. Dann gilt
\begin{equation*}
\mathbb{P}_{0} [S_{\lbrace 0 \rbrace} = n] = u_{n-1} - u_{n}
\end{equation*}
Daraus folgt
\begin{equation*}
\mathbb{P}_{0} [S_{\lbrace 0 \rbrace} < \infty] = \lim_{N \to \infty} \sum_{n=1}^{N} \mathbb{P}_{0} [S_{\lbrace 0 \rbrace} = n] = \lim_{N \to \infty}(1-u_{N}) = 1 - \lim_{N \to \infty } \prod_{n=0}^{N-1}  (1 - p_{n})
\end{equation*}
\underline{Beh.}: Falls $p_{i} \in (0,1)$ für alle $i \in \mathbb{N}_{0}$ so gilt
\begin{equation*}
\lim_{N \to \infty} \prod_{n=0}^{N} (1-p_{n}) = 0 \quad \Leftrightarrow \quad \sum_{n=0}^{\infty} p_{n} = \infty
\end{equation*}
$"\Leftarrow"$ Da $e^{-x} \geq 1-x$ für alle $x \geq 0$, folgt
\begin{equation*}
0 \leq \lim_{N \to \infty} \prod_{n=0}^{N} (1-p_{n}) \leq \lim_{N \to \infty} exp(-\sum_{n=0}^{N}p_{n}) \stackrel{\sum_{n=0}^{\infty}p_{n} = \infty}{=} 0
\end{equation*}
$"\Rightarrow"$ $\dashuline{zu \: zeigen}:$ Für jedes $m \in \mathbb{N}_{0}$ gilt $\prod_{n=m}^{m+N}(1-p_{n}) \geq 1 - \sum_{n=m}^{m+N}p_{n}$, $\forall N \in \mathbb{N}_{0}$
\\
\\
\textbf{IA} $N=0$: $\checkmark$
\\
\\
\textbf{IS} $N \to N+1$: Es gilt nun aber
\begin{equation*}
\prod_{n=m}^{m+N+1} (1-p_{n}) \stackrel{\mathrm{IV}}{\geq} (1 - \sum_{n=m}^{m+N}p_{n})(1-p_{m+N+1}) = 1 - p_{m+N+1} - (\underbrace{1-p_{m+N+1}}_{\leq 1}) \cdot \sum_{n=m}^{m+N}p_{n} \geq 1 - \sum_{n=m}^{m+N+1} p_{n}
\end{equation*}
Angenommen $\sum_{n=0}^{\infty}p_{n} < \infty$. Dann existiert ein $m \in \mathbb{N}$ so, dass $0 < \sum_{n=m}^{\infty} p_{n} < 1$.
\\
Daraus folgt
\begin{equation*}
0 = \lim_{N \to \infty} \prod_{n=0}^{N} (1-p_{n}) = \prod_{n=0}^{m-1} (1-p_{n})  \lim_{N \to \infty} \prod_{n=m}^{N} (1-p_{n}) \geq \prod_{n=0}^{m-1} (1-p_{n}) \cdot (1 - \underbrace{\sum_{n=m}^{\infty} p_{n}}_{<1}) > 0 \: \: \lightning
\end{equation*}
Folglich ist $\sum_{n=0}^{\infty} p_{n} = \infty$.

\textbf{Beispiel 2.11}[Einfache Irrfahrt auf $\mathbb{Z}$]
Sei $(X_{n})_{n \in \mathbb{N}_{0}}$ eine Markovkette auf $E=\mathbb{Z}$ mit folgendem Übergangsgraphen:
\begin{figure}[H].
\centering
\includegraphics[scale=0.4]{Beispiel_Einfache_Irrfahrt_auf_Z}
\caption{Einfache Irrfahrt auf $\mathbb{Z}$}
\end{figure}
\noindent
Dann gilt $p_{2n+1}(0,0) = 0$ für jedes $n \in \mathbb{N}_{0}$ und $p_{2n}(0,0) = \binom{2n}{n} p^{n} (1-p)^{n}$ für jedes $n \in \mathbb{N}$. Aus der Stirlingformel
\begin{equation*}
n! \sim \sqrt{2 \pi n} \cdot n^{n} e^{-n} \qquad (a_{n} \sim b_{n} :\Leftrightarrow \lim_{n \to \infty} \dfrac{a_{n}}{b_{n}} = 1)
\end{equation*}
folgt dann
\begin{equation*}
p_{2n}(0,0) = \dfrac{(2n)!}{(n!)^{2}} \cdot p^{n}(1-p)^{n} \sim \dfrac{\sqrt{4 \pi n}}{2 \pi n} \dfrac{(2n)^{2n}}{n^{2n}} \cdot p^{n} (1-p)^{n} = \dfrac{1}{\sqrt{\pi n}} (4p(1-p))^{n}
\end{equation*}
\underline{1.Fall}: $p=\dfrac{1}{2}$ $\:$  $\Rightarrow$ $\:$  $p_{2n}(0,0)$ $\sim$ $\dfrac{1}{\sqrt{\pi n}}$ $\:$  $\Rightarrow$ $\:$ $\exists$ $n_{0} \in \mathbb{N} \: : \: p_{2n}(0,0) \geq \dfrac{1}{2\sqrt{n}}$ $\:$  $\forall n \geq n_{0}$
\\
Also
\begin{equation*}
\sum_{n=1}^{\infty} p_{n}(0,0) \geq \sum_{n=n_{0}}^{\infty}p_{2n}(0,0) \geq \dfrac{1}{2} \sum_{n=n_{0}}^{\infty}\dfrac{1}{\sqrt{n}} = \infty
\end{equation*}
Aus Satz $\ref{alternative Chrakterisierung von rekurrent/transient}$ folgt somit, dass $x=0$ rekurrent ist.
\\
\\
\underline{2.Fall}: $p \neq \dfrac{1}{2}$ $\:$  $\Rightarrow$ $\:$ $4p(1-p) =: r < 1$ $\:$  $\Rightarrow$ $\:$ $\exists$ $n_{0} \in \mathbb{N} \: : \: p_{2n}(0,0) \leq r^{n}$ $\:$  $\forall n \geq n_{0}$
\\
Also,
\begin{equation*}
\sum_{n=1}^{\infty} p_{n}(0,0) = \sum_{n=1}^{\infty} p_{2n}(0,0) \leq n_{0} + \sum_{n=n_{0}}^{\infty} r^{n} < \infty
\end{equation*}
Aus Satz $\ref{alternative Chrakterisierung von rekurrent/transient}$ folgt somit, dass $x=0$ transient ist.Sei $(X_{n})_{n \in \mathbb{N}_{0}}$ eine Markovkette mit Zustandsraum $E = \mathbb{N}_{0}$, dessen Übergangsmatrix $P = (p(x,y))_{x,y \in E}$ durch folgenden Übergangsgraphen beschrieben wird
\begin{figure}[H].
\centering
\includegraphics[scale=0.5]{Beispiel_Kartenhausprozess}
\caption{Übergangsgraph des Kartenhausprozesses}
\end{figure}
\noindent
\textcolor{red}{Frage?} Unter welchen Bedingungen an $(p_{i})_{i \in \mathbb{N}_{0}}$ ist der Zustand $x=0$ rekurrent?
\\
\\
Da es für jedes $n \in \mathbb{N}$ genau einen Pfad gibt, der bei Start in $x=0$ nach genau n-Schritten wieder in 0 trifft (nämlich $(X_{0},X_{1},...,X_{n}) = (0,1,2,...,n-1,0)$) folgt
\begin{equation*}
\mathbb{P}_{0} [S_{\lbrace 0 \rbrace} = n] = (1-p_{0}) \cdot ... \cdot (1-p_{n-2}) \cdot p_{n-1} \: , \quad n \in \mathbb{N}
\end{equation*} 
Setze nun $u_{0} := 1$ und $u_{n} = (1-p_{0}) \cdot ... \cdot (1-p_{n-1})$, $\:$ $n \in \mathbb{N}$. Dann gilt
\begin{equation*}
\mathbb{P}_{0} [S_{\lbrace 0 \rbrace} = n] = u_{n-1} - u_{n}
\end{equation*}
Daraus folgt
\begin{equation*}
\mathbb{P}_{0} [S_{\lbrace 0 \rbrace} < \infty] = \lim_{N \to \infty} \sum_{n=1}^{N} \mathbb{P}_{0} [S_{\lbrace 0 \rbrace} = n] = \lim_{N \to \infty}(1-u_{N}) = 1 - \lim_{N \to \infty } \prod_{n=0}^{N-1}  (1 - p_{n})
\end{equation*}
\underline{Beh.}: Falls $p_{i} \in (0,1)$ für alle $i \in \mathbb{N}_{0}$ so gilt
\begin{equation*}
\lim_{N \to \infty} \prod_{n=0}^{N} (1-p_{n}) = 0 \quad \Leftrightarrow \quad \sum_{n=0}^{\infty} p_{n} = \infty
\end{equation*}
$"\Leftarrow"$ Da $e^{-x} \geq 1-x$ für alle $x \geq 0$, folgt
\begin{equation*}
0 \leq \lim_{N \to \infty} \prod_{n=0}^{N} (1-p_{n}) \leq \lim_{N \to \infty} exp(-\sum_{n=0}^{N}p_{n}) \stackrel{\sum_{n=0}^{\infty}p_{n} = \infty}{=} 0
\end{equation*}
$"\Rightarrow"$ $\dashuline{zu \: zeigen}:$ Für jedes $m \in \mathbb{N}_{0}$ gilt $\prod_{n=m}^{m+N}(1-p_{n}) \geq 1 - \sum_{n=m}^{m+N}p_{n}$, $\forall N \in \mathbb{N}_{0}$
\\
\\
\textbf{IA} $N=0$: $\checkmark$
\\
\\
\textbf{IS} $N \to N+1$: Es gilt nun aber
\begin{equation*}
\prod_{n=m}^{m+N+1} (1-p_{n}) \stackrel{\mathrm{IV}}{\geq} (1 - \sum_{n=m}^{m+N}p_{n})(1-p_{m+N+1}) = 1 - p_{m+N+1} - (\underbrace{1-p_{m+N+1}}_{\leq 1}) \cdot \sum_{n=m}^{m+N}p_{n} \geq 1 - \sum_{n=m}^{m+N+1} p_{n}
\end{equation*}
Angenommen $\sum_{n=0}^{\infty}p_{n} < \infty$. Dann existiert ein $m \in \mathbb{N}$ so, dass $0 < \sum_{n=m}^{\infty} p_{n} < 1$.
\\
Daraus folgt
\begin{equation*}
0 = \lim_{N \to \infty} \prod_{n=0}^{N} (1-p_{n}) = \prod_{n=0}^{m-1} (1-p_{n})  \lim_{N \to \infty} \prod_{n=m}^{N} (1-p_{n}) \geq \prod_{n=0}^{m-1} (1-p_{n}) \cdot (1 - \underbrace{\sum_{n=m}^{\infty} p_{n}}_{<1}) > 0 \: \: \lightning
\end{equation*}
Folglich ist $\sum_{n=0}^{\infty} p_{n} = \infty$.

\textbf{Beispiel 2.11}[Einfache Irrfahrt auf $\mathbb{Z}$]
Sei $(X_{n})_{n \in \mathbb{N}_{0}}$ eine Markovkette auf $E=\mathbb{Z}$ mit folgendem Übergangsgraphen:
\begin{figure}[H].
\centering
\includegraphics[scale=0.4]{Beispiel_Einfache_Irrfahrt_auf_Z}
\caption{Einfache Irrfahrt auf $\mathbb{Z}$}
\end{figure}
\noindent
Dann gilt $p_{2n+1}(0,0) = 0$ für jedes $n \in \mathbb{N}_{0}$ und $p_{2n}(0,0) = \binom{2n}{n} p^{n} (1-p)^{n}$ für jedes $n \in \mathbb{N}$. Aus der Stirlingformel
\begin{equation*}
n! \sim \sqrt{2 \pi n} \cdot n^{n} e^{-n} \qquad (a_{n} \sim b_{n} :\Leftrightarrow \lim_{n \to \infty} \dfrac{a_{n}}{b_{n}} = 1)
\end{equation*}
folgt dann
\begin{equation*}
p_{2n}(0,0) = \dfrac{(2n)!}{(n!)^{2}} \cdot p^{n}(1-p)^{n} \sim \dfrac{\sqrt{4 \pi n}}{2 \pi n} \dfrac{(2n)^{2n}}{n^{2n}} \cdot p^{n} (1-p)^{n} = \dfrac{1}{\sqrt{\pi n}} (4p(1-p))^{n}
\end{equation*}
\underline{1.Fall}: $p=\dfrac{1}{2}$ $\:$  $\Rightarrow$ $\:$  $p_{2n}(0,0)$ $\sim$ $\dfrac{1}{\sqrt{\pi n}}$ $\:$  $\Rightarrow$ $\:$ $\exists$ $n_{0} \in \mathbb{N} \: : \: p_{2n}(0,0) \geq \dfrac{1}{2\sqrt{n}}$ $\:$  $\forall n \geq n_{0}$
\\
Also
\begin{equation*}
\sum_{n=1}^{\infty} p_{n}(0,0) \geq \sum_{n=n_{0}}^{\infty}p_{2n}(0,0) \geq \dfrac{1}{2} \sum_{n=n_{0}}^{\infty}\dfrac{1}{\sqrt{n}} = \infty
\end{equation*}
Aus Satz $\ref{alternative Chrakterisierung von rekurrent/transient}$ folgt somit, dass $x=0$ rekurrent ist.
\\
\\
\underline{2.Fall}: $p \neq \dfrac{1}{2}$ $\:$  $\Rightarrow$ $\:$ $4p(1-p) =: r < 1$ $\:$  $\Rightarrow$ $\:$ $\exists$ $n_{0} \in \mathbb{N} \: : \: p_{2n}(0,0) \leq r^{n}$ $\:$  $\forall n \geq n_{0}$
\\
Also,
\begin{equation*}
\sum_{n=1}^{\infty} p_{n}(0,0) = \sum_{n=1}^{\infty} p_{2n}(0,0) \leq n_{0} + \sum_{n=n_{0}}^{\infty} r^{n} < \infty
\end{equation*}
Aus Satz $\ref{alternative Chrakterisierung von rekurrent/transient}$ folgt somit, dass $x=0$ transient ist.

\textbf{Beispiel 2.11}[einefache, symmterische Irrfahrt auf $\mathbb{Z}^{2}$]
Sei $(X_{n})_{n \in \mathbb{N}_{0}}$ eine Markovkette auf $E=\mathbb{Z}^2$ mit $p(x,y)=\dfrac{1}{4} \mathbbm{1}_{\vert \vert x-y \vert \vert = 1}$. Zunächst einmal ist $p_{2n+1}(0,0)=0$ für alle $n \mathbb{N}_{0}$. Um in 2n Schritten nach $x=0$ zurückzukehren muss $(X_{n})_{n \in \mathbb{N}_{0}}$ gleich oft (k-Mal) nach rechts bzw. links und gleich oft ((n-k)-Mal) nach oben bzw. unten gelaufen sein. Daraus folgt
\begin{equation*}
p_{2n}(0,0) = 4^{-2n} \sum_{k=0}^{n} \dfrac{(2n)!}{(k!)^{2}((n-k)!)^{2}} = 4^{-2n} \binom{2n}{n} \underbrace{\sum_{k=0}^{n} \binom{n}{k} \binom{n}{n-k}}_{= \binom{2n}{n}} = ({\underbrace{\binom{2n}{n} 2^{-n}}_{\sim \dfrac{1}{\sqrt{\pi n}}}})^{2} \sim \dfrac{1}{\pi n}
\end{equation*}
Also existiert wiederum ein $n_{0} \in \mathbb{N}$ mit $p_{2n}(0,0) \geq \dfrac{1}{4n}$ für alle $n \geq n_{0}$. Daraus folgt 
\begin{equation*}
\sum_{n=1}^{\infty} p_{n}(0,0) \geq \sum_{n = n_{0}}^{\infty} p_{2n}(0,0) \geq \dfrac{1}{4} \sum_{n = n_{0}}^{\infty} \dfrac{1}{n} = \infty
\end{equation*}
$\Rightarrow$ x=0 ist rekurrent.