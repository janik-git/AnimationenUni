\mbox{}
\begin{itemize}
\item[a)] Ein Zustand $y \in E$ heißt erreichbar von $x \in E$ $(x \rightarrow y)$, falls ein $n \in \mathbb{N}_{0}$ existiert mit $p_{n}(x,y)>0$
\item[b)] Die Zustände $x,y \in E$ kommunizieren $(x\leftrightarrow y)$, falls $x \rightarrow y$ und $y \rightarrow x$. 
\item[c)] Eine nichtleere Teilmenge $\emptyset \neq K \subseteq E$ heißt kommunizierende Klasse, falls
\begin{itemize}
\item[(i)] $x \leftrightarrow y$ für alle $x,y \in K$
\item[(ii)] aus $x \in K$ und $y \in E$ mit $x \leftarrow y$ folgt $y \in K$ $\quad (Abgeschlossenheit)$
\end{itemize}
\item[d)] Ist $x \in E$ Element einer kommunizierenden Klasse, so heißt x wesentlich(sonst unwesentlich).
\end{itemize}