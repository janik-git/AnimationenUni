\label{Bemerkung zu Teilern}
Sei $D \subseteq \mathbb{Z}$. Dann ist $d \in \mathbb{N}$ ein gemeinsamer Teiler von D (Schreibweise: $d \: \vert \: D$), falls $\dfrac{x}{d} \in \mathbb{Z}$ für alle $x \in D$. Ist $D = \lbrace 0 \rbrace$, so gilt $d \: \vert \: D$ für alle $d \in \mathbb{N}$, also $ggT(D)=\infty$. Falls $D \neq \lbrace 0 \rbrace$, so gilt
\begin{equation*}
d \: \vert \: D \quad \Rightarrow \quad d \leq \min \lbrace \vert x \vert \: : \: x \in D\setminus \lbrace 0 \rbrace \rbrace
\end{equation*}
Insbesondere ist
\begin{equation*}
ggT(D) = \min \lbrace \vert x \vert \: : \: x \in D\setminus \lbrace 0 \rbrace \rbrace \quad \Leftrightarrow \quad \lbrace ggT(D), -ggT(D) \rbrace \cap D \neq \emptyset 
\end{equation*}