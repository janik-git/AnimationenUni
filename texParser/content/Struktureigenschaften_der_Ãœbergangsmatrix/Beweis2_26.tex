Zunächst einmal gilt für jedes $x \in E$
\begin{equation*}
\sum_{y \in E}G(x,y) = \sum_{n=0}^{\infty} \sum_{y \in E} p_{n}(x,y) = \sum_{n=0}^{\infty} 1 = \infty
\end{equation*}
Da E endlich ist, gibt es folglich ein $y \in E$ mit $G(x,y) = \infty$. Da E aufgrund der Irreduzibilität nur aus einer kommunizierenden Klasse besteht, ist insbesondere $y \rightarrow x$. Folglich existiert ein $m \in \mathbb{N}$ mit $p_{m}(x,y)>0$. Aus der Chapman-Kolmogorov-Gleichung folgt zudem $p_{m+n}(x,x) \geq p_{n}(x,y) \cdot p_{m}(y,x)$. Also,
\begin{equation*}
G(x,x) \geq \sum_{n=0}^{\infty} p_{n}(x,y)p_{m}(y,x) = \underbrace{p_{m}(y,x)}_{>0} \cdot G(x,y) = \infty
\end{equation*}
Somit ist x nach Satz $\ref{alternative Chrakterisierung von rekurrent/transient}$ rekurrent. Aus Satz $\ref{rekkurent und x -> y so gilt y -> x und y rekurrent}$ folgt dann aber, dass jeder Zustand in E rekurrent ist. Angenommen $x \in E$ wäre nullrekurrent. Dann folgt aus Satz $\ref{x und y selbe Periode, x transient y auch x nullrekurrent y auch}$, dass jeder Zustand nullrekurrent ist. Aber dann folgt aus Korollar $\ref{nullrekurrent und limes}$
\begin{equation*}
1 = \lim_{n \to \infty} \sum_{y \in E} p_{n}(x,y) \stackrel{\vert E \vert < \infty}{=} 0 \: \: \lightning 
\end{equation*}
$\Rightarrow$ alle Zustände in E sind positiv rekurrent.  