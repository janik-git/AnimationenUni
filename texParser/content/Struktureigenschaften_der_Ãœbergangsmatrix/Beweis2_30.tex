Da nach Vorraussetzungen $(X_{n})_{n \in \mathbb{N}_{0}}$ irreduzibel ist, gibt es nach Definition nur eine kommunizierende Klasse.
\begin{itemize}
\item[a)] Für $x \in A$ gilt $S_{\lbrace x \rbrace}(\omega) \geq S_{A}(\omega)$ für alle $\omega \in \Omega$. Daraus folgt
\begin{equation*}
0<\mathbb{P}_{x}[S_{A}=\infty] = \mathbb{P}_{x}[S_{\lbrace x \rbrace} \geq S_{A}, S_{A} = \infty] \leq \mathbb{P}_{x}[S_{\lbrace x \rbrace} = \infty] \: \Leftrightarrow \: \mathbb{P}_{x}[S_{\lbrace x \rbrace} < \infty] < 1
\end{equation*}
Also ist jeder Zustand x transient. Der Satz $\ref{x und y selbe Periode, x transient y auch x nullrekurrent y auch}$ impliziert nun, dass jeder Zustand $y \in E$ transient ist. 
\item[b)] Bezeichne wieder mit $S_{A}^{k}$ die k-te Treffzeit der Menge A, d.h.
\begin{equation*}
S_{A}^{0} := 0 \quad und \quad S_{A}^{k} := \inf \lbrace n> S_{A}^{k-1} \: : \: X_{n} \in A \rbrace, \quad k \in \mathbb{N}
\end{equation*}
Da $\mathbb{P}_{x}[S_{A} < \infty] = 1$ für alle $x \in A$, so folgt für jedes $n \in \mathbb{N}$
\begin{equation*}
\mathbb{P}_{x} [S_{A}^{n} < \infty] 
\end{equation*}
\begin{equation*}
= \sum_{y \in A} \mathbb{P}_{x}[S_{A}^{n} < \infty \: | \: S_{A}^{n-1} < \infty, X_{S_{A}^{n}}=y]\mathbb{P}_{x} [S_{A}^{n-1} < \infty, X_{S_{A}^{n-1}} = y] 
\end{equation*}
\begin{equation*}
\stackrel{\mathrm{Satz} \: \ref{starke Markoveigenschaft}}{=} \sum_{y \in A} \underbrace{\mathbb{P}_{y}[S_{A} < \infty]}_{= 1} \mathbb{P}_{x} [S_{A}^{n-1} < \infty, X_{S_{A}^{n-1}} = y]
\end{equation*}
\begin{equation*}
= \mathbb{P}_{x}[S_{A}^{n-1} < \infty]
\end{equation*}
Induktiv ergibt sich daraus, dass $\mathbb{P}_{x}[S_{A}^{k} < \infty] = 1$ für alle $n \in \mathbb{N}$ und $x \in A$.
\\
Da $(X_{n})_{n \in \mathbb{N}_{0}}$ irreduzibel ist, gilt für alle $x \in A$ und $y \in E \setminus A$, dass $x \leftrightarrow y$, d.h.
\begin{equation*}
\mathbb{P}_{x}[S_{\lbrace y \rbrace} = \infty] < 1
\end{equation*}
Folglich existiert zu jedem $x \in A$ ein $N_{x} \in \mathbb{N}$ und $\epsilon_{x} > 0$ mit 
\begin{equation*}
\mathbb{P}_{x}[n < S_{\lbrace y \rbrace}] \leq 1 - \epsilon_{x} \qquad \forall \: n \geq N_{x}
\end{equation*}
Setze $N := \max \lbrace N_{x} \: : \: x \in A \rbrace$ und $\epsilon := \min \lbrace \epsilon_{x} \: : \: x \in A \rbrace$. Da A endlich ist, gilt $N < \infty, \: \epsilon > 0 $ und
\begin{equation*}
\mathbb{P}_{x}[n < S_{\lbrace y \rbrace}] \leq 1 - \epsilon \qquad \forall n \geq N, \: x \in A
\end{equation*}
Da $S_{A}^{n} \geq n$, folgt somit
\begin{equation*}
\mathbb{P}_{x}[S_{A}^{n} < S_{\lbrace y \rbrace}] \leq 1 - \epsilon \qquad \forall n \geq N, \: x \in A
\end{equation*}
Zudem gilt für alle $k \in \mathbb{N}$
\begin{equation*}
\mathbb{P}_{x}[S_{A}^{kN} < S_{\lbrace y \rbrace}] = 
\end{equation*}
\begin{equation*}
\sum_{z \in A} \mathbb{P}_{x}[S_{A}^{kN} < S_{\lbrace y \rbrace} \: | \: S_{A}^{(k-1)N} < S_{\lbrace y \rbrace}, X_{S_{A}^{(k-1)N}} = z] \mathbb{P}_{x}[ S_{A}^{(k-1)N} < S_{\lbrace y \rbrace}, X_{S_{A}^{(k-1)N}} = z]
\end{equation*}
\begin{equation*}
\stackrel{\mathrm{Satz} \: \ref{starke Markoveigenschaft}}{=} \sum_{z \in A} \underbrace{\mathbb{P}_{z}[S_{A}^{N} < S_{\lbrace y \rbrace}]}_{\leq 1-\epsilon} \mathbb{P}_{x}[ S_{A}^{(k-1)N} < S_{\lbrace y \rbrace}, X_{S_{A}^{(k-1)N}} = z]
\end{equation*}
\begin{equation*}
\leq (1- \epsilon)\mathbb{P}_{x}[ S_{A}^{(k-1)N} < S_{\lbrace y \rbrace}].
\end{equation*}
Daraus ergibt sich induktiv, dass
\begin{equation*}
\mathbb{P}_{x}[S_{A}^{kN} < S_{\lbrace y \rbrace}] \leq (1- \epsilon)^{k} \qquad \forall \: k \in \mathbb{N}, \: x \in A
\end{equation*}
Somit erhält man 
\begin{equation*}
\mathbb{P}_{x}[S_{\lbrace y \rbrace} = \infty] = \limsup_{k \to \infty} \mathbb{P}_{x}[S_{A}^{kN} < \infty, S_{\lbrace y \rbrace} = \infty]
\end{equation*}
\begin{equation*}
\leq \limsup_{k \to \infty} \mathbb{P}_{x}[S_{A}^{kN} < S_{\lbrace y \rbrace}] = \limsup_{k \to \infty} (1 - \epsilon)^{k} = 0
\end{equation*}
Also,
\begin{equation*}
\mathbb{P}_{x}[S_{\lbrace y \rbrace} < \infty] = 1
\end{equation*}
Angenommen y wäre transient. Dann folgt aus Satz $\ref{irreduzibel, y rekurrent -> Px=1 , y transient -> Px<1 }$, dass
\begin{equation*}
\mathbb{P}_{x}[S_{\lbrace y \rbrace} < \infty] < 1 \quad \lightning
\end{equation*}
Folglich ist y rekurrent. Aus Satz $\ref{rekkurent und x -> y so gilt y -> x und y rekurrent}$ folgt dann aber, dass jeder Zustand $y \in E$ rekurrent ist.
\end{itemize}