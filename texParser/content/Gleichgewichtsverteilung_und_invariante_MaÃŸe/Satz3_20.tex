\label{Satz 3.6}
Sei $(X_{n})_{n \in \mathbb{N}_{0}}$ eine $(\nu,P)$-Markovkette mit Zustandsraum E und $\emptyset \neq K \subseteq E$ eine kommunizierende Klasse. Dann gelten folgende Aussagen:
\begin{itemize}
\item[a)] Es gibt ein $\pi \in Inv(P)$ mit $\sum_{x \in K} \pi(x) = 1$ genau dann, wenn K positiv rekurrent ist. Insbesondere ist
\begin{equation*}
 \pi(x) = \dfrac{1}{\mathbb{E}_{x}[S_{\lbrace x \rbrace}]} \qquad \forall \: x \in K
\end{equation*} 
\item[b)] Wenn $K \neq E$ transient ist und $\pi \in Inv(p)$, so ist $\pi(x) = 0$ für alle $x \in K$ 
\end{itemize}