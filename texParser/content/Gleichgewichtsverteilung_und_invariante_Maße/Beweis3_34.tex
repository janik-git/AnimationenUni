Da $P$ irreduzibel ist, ist nach Bemerkung $\ref{Auflistende Bemerkung zu invarianten Maßen}$ c) $\pi(x) > 0$ für alle $x \in E$. Somit sind die Matrixeinträge von $P^{*}$ wohldefiniert. Zudem gilt
\begin{equation*}
\sum_{y \in E} p^{*}(x,y) = \dfrac{1}{\pi (x)} \sum_{y \in E} \pi (y) p(y,x) = 1 \qquad \forall \: x \in E
\end{equation*}
d.h. $P^{*}$ ist eine stochastische Matrix. Aufgrund von Satz $\ref{Besitzen Markovketten die Markoveigenschaft}$ (i) genügt es nun
\\
\\
\dashuline{zu zeigen}: $\: \forall \: n \in \lbrace 0,1,...,N \rbrace$ und $y_{0},...,y_{n} \in E$ gilt
\begin{equation*}
\mathbb{P}_{\pi} [Y_{0} = y_{0},...,Y_{n}=y_{n}] = \pi(y_{0})p^{*}(y_{0},y_{1}) \cdot ... \cdot p^{*}(y_{n-1},y_{n})
\end{equation*}
Für $n \in \lbrace 0,1,...,N \rbrace$ und $y_{0},...,y_{n} \in E$ betrachte nun
\begin{equation*}
\mathbb{P}_{\pi} [Y_{0} = y_{0},...,Y_{n}=y_{n}] = \mathbb{P}_{\pi} [X_{N} = y_{0},...,X_{N-n}=y_{n}] 
\end{equation*}
\begin{equation*}
= \mathbb{P}_{\pi}[X_{N-n} = y_{n}] \mathbb{P}_{\pi} [X_{N} = y_{0},...,X_{N-n+1}=y_{n-1} \: | \: X_{N-n}=y_{n}] 
\end{equation*}
\begin{equation*}
\stackrel{\mathrm{Satz} \: \ref{vorangegangene und zukünftige Ereignisse}}{=} \mathbb{P}_{\pi}[X_{N-n} = y_{n}] \mathbb{P}_{Y_{n}} [X_{n} = y_{0},...,X_{1}=y_{n-1}] 
\end{equation*}
\begin{equation*}
\stackrel{\mathrm{Satz} \: \ref{"Satz 1.8"}}{=} \underbrace{(\pi P^{N-n})}_{\pi}(y_{n}) \mathbb{P}_{Y_{n}} [X_{1} = y_{n-1},...,X_{n}=y_{0}] 
\end{equation*}
\begin{equation*}
\stackrel{\mathrm{Satz} \: \ref{Besitzen Markovketten die Markoveigenschaft}}{=} \pi(y_{n}) p(y_{n},y_{n-1}) \cdot ... \cdot p(y_{1},y_{0})
\end{equation*}
\begin{equation*}
= \pi(y_{0})p^{*}(y_{0},y_{1}) \cdot ... \cdot p^{*}(y_{n-1},y_{n})
\end{equation*}
Somit ist nach Satz $\ref{Besitzen Markovketten die Markoveigenschaft}$ (i) $(Y_{n})_{0 \leq n \leq N}$ eine $(\pi,P^{*})$-Markovkette.