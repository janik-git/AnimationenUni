\label{aufzählungen existenz von invarianten Maßen}
Sei $(X_{n})_{n \in \mathbb{N}_{0}}$ eine $(\nu,P)$-Markovkette mit Zustandsraum E und $\emptyset \neq K \subseteq E$ eine rekurrente, kommunizierende Klasse. Für $x \in K$ definiere
\begin{equation*}
\mu_{x}(y) := \mu_{x}[\lbrace y \rbrace] := \mathbb{E}_{x}[\sum_{n=0}^{S_{\lbrace x \rbrace} - 1} \mathbbm{1}_{X_{n} = y}] \quad , \quad y \in E
\end{equation*}
wobei $S_{\lbrace x \rbrace} := \inf \lbrace n \in \mathbb{N} \: : \: X_{n} = x  \rbrace$ die erste Treffzeit des Zustandes $x \in E$ sei.
\begin{itemize}
\item[a)] Dann gilt für alle $x,y \in K$ mit $x \neq y$
\begin{equation*}
\mathbb{P}_{x}[S_{\lbrace y \rbrace} < S_{\lbrace x \rbrace}] > 0 \quad und \quad \mathbb{E}_{x}[\sum_{n=0}^{S_{\lbrace x \rbrace} - 1} \mathbbm{1}_{X_{n} = y}] = \dfrac{\mathbb{P}_{x}[S_{\lbrace y \rbrace} < S_{\lbrace x \rbrace}]}{\mathbb{P}_{y}[S_{\lbrace x \rbrace} < S_{\lbrace y \rbrace}]}
\end{equation*} 
Insbesondere ist $\mu_{x}(x)=1, \: \mu_{x}(y) =0$ für alle $y \in K^{C}$ und $\mu_{x}(y) \in (0,\infty)$ für alle $y \in K$.
\item[b)] Für jedes $x \in K$ ist $\mu_{x}$ ein invariantes Maß bzgl. $P$.
\item[c)] Ist $\lambda$ ein invariantes Maß bzgl. $P$ mit $\lambda(x) =1$ für ein $x \in K$ und $\lambda(y) = 0$ für alle $y \in K^{C}$, so gilt $\lambda = \mu_{x}$ 
\end{itemize}