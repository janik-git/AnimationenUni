\mbox{}
\begin{itemize}
\item[a)] Definiere $\bar{P} = (\bar{p}(x,y))_{x,y \in E}$ durch $\bar{p}(x,y) := \sum_{n=1}^{\infty} 2^{-n} p_{n}(x,y) \:, \quad x,y \in E$. Dann ist $\bar{P}$ eine stochastische Matrix, denn für alle $x \in E$ gilt
\begin{equation*}
\sum_{y \in E} \bar{p} (x,y) \stackrel{\mathrm{Fubini}}{=} \sum_{n=1}^{\infty} 2^{-n} \sum_{y \in E} p_{n}(x,y) = \sum_{n=1}^{\infty} 2^{-n} = 1
\end{equation*}
Da P nach Voraussetzungen irreduzibel ist, folgt $\bar{p}(x,y)>0$ für alle $x,y \in E$. Angenommen es gäbe zwei Gleichgewichtsverteilungen $\pi_{1},\pi_{2} \in Inv(P)$ mit $\pi_{1} \neq \pi_{2}$. Da
\begin{equation*}
(\pi_{i} \bar{P})(x) = \sum_{y \in E} \pi_{i}(y) \bar{p}(y,x) \stackrel{\mathrm{Fubini}}{=} \sum_{n=1}^{\infty} 2^{-n} \sum_{y \in E} \pi_{i} p(y,x) = \pi_{i}(x) \sum_{n=1}^{\infty} 2^{-n} = \pi_{i}(x)
\end{equation*} 
für alle $x \in E$ und $i \in \lbrace 1,2 \rbrace$, ist $\pi_{1}, \pi_{2} \in Inv(\bar{P})$. Betrachte nun das signierte Maß
\begin{equation*}
\bar{\pi} := \pi_{1} - \pi_{2}.
\end{equation*}
Dann gilt $\bar{\pi} \bar{P} = \pi_{1} \bar{P} - \pi_{2} \bar{P} = \pi_{1} - \pi_{2} = \bar{\pi}$. Da $\bar{\pi} \neq 0$ und $\bar{\pi}[E]=0$ existieren $x,y \in E$ mit $\bar{\pi}(x) > 0$ und $\bar{\pi}(y) < 0$. Weiterhin gilt
\begin{equation*}
\sum_{z \in E} \vert \bar{\pi}(z) \vert = \sum_{z \in E} \vert (\bar{\pi}\bar{P})(z) \vert
\end{equation*}
\begin{equation*}
= \sum_{z \in E}\vert \underbrace{\bar{\pi}(x)\bar{p}(x,z)}_{>0} + \underbrace{\bar{\pi}(y)\bar{p}(y,z)}_{<0} +  \sum_{\substack{ z' \in E \\ z' \neq x,y } }\bar{\pi}(z')\bar{p}(z',z) \vert
\end{equation*}
\begin{equation*}
< \sum_{z \in E} \sum_{z' \in E} \vert \bar{\pi}(z') \vert \bar{p}(z',z)
\end{equation*}
\begin{equation*}
= \sum_{z' \in E} \vert \bar{\pi}(z') \vert \quad \lightning
\end{equation*}
Folglich ist $\bar{\pi}=0$, d.h. $\pi_{1} = \pi_{2}$.
\item[b)] Angenommen $Inv(P) \neq \emptyset$, d.h. es gibt eine Gleichgewichtsverteilung $\pi$. Nach Voraussetzung ist jeder Zustand $y \in E$ transient. Also folgt aus dem Korollar $\ref{transienter Zustand dann lim n -> unendl. pn(x,y) = 0}$ und dem Satz von Lebesgue
\begin{equation*}
0 = \sum_{x \in E} \pi (x) \lim_{n \to \infty}p_{n}(x,y) = \lim_{n \to \infty} \sum_{x \in E} \pi(x) p_{n}(x,y) = \pi(y) \qquad \forall \: y \in E
\end{equation*}
\mbox{}
\\
Also,
\begin{equation*}
\sum_{y \in E} \pi (y) = 0 \neq 1 \quad \lightning
\end{equation*}
Folglich gibt es keine Gleichgewichtsverteilung.
\end{itemize}