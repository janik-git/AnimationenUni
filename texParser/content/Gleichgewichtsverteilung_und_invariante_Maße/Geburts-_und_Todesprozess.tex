Sei $(X_{n})_{n \in \mathbb{N}_{0}}$ eine Markovkette auf $E = \mathbb{N}_{0}$, dessen Übergangsmatrix $P$ durch folgenden Übergangsgraphen beschrieben wird
\begin{figure}[H].
\centering
\includegraphics[scale=0.34]{Geburts- und Todesprozess}
\caption{Übergangsgraph eines Geburts- und Todesprozess}
\end{figure}
\noindent
wobei angenommen sei, dass $q_{x} > 0$ für alle $x \in \mathbb{N}$. Setze
\begin{equation*}
\pi(0) := 1 \quad und \quad \pi(x) = \prod_{y=1}^{x} \dfrac{p_{y-1}}{q_{y}}, \quad x \in \mathbb{N}
\end{equation*} 
Dann gilt
\begin{equation*}
\pi(x-1)p(x-1,x) = \pi(x-1)p_{x-1} = \pi(x-1) \dfrac{p_{x-1}}{q_{x}}p(x,x-1) = \pi(x)p(x,x-1)
\end{equation*}
Folglich ist $\pi$ reversible bzgl. $P$ und insbesondere ein invariantes Maß. Falls zudem gilt, dass
\begin{equation*}
\sum_{x \in E} \pi(x) = \sum_{x \in E} \prod_{y=1}^{x} \dfrac{p_{y-1}}{q_{y}} < \infty
\end{equation*}
so lässt sich $\pi$ zu einer Gleichverteilung normieren.