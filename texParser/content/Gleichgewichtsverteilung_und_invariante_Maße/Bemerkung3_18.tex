
\label{Auflistende Bemerkung zu invarianten Maßen}
\mbox{}
\begin{itemize}
\item[a)] Ein invariantes Maß $\pi : \: E \to [0,1]$ ist als Zeilenvektor $(\pi \in [0,\infty]^{E})$ aufgefasst ein (nichtnegativer) Linkseigenvektor von $P$ zum Eigenwert 1.
\item[b)] Ist $\vert E \vert < \infty$, so kann jedes invariantes Maß zu einer Gleichgewichtsverteilung normiert werden.
\item[c)] Ist $\pi$ ein invariantes Maß bzgl. $P$, so gilt $\pi = \pi P^{n}$ für jedes $n \in \mathbb{N}_{0}$. Falls $P$ zudem irreduzibel und $\pi \neq 0$ ist, so folgt
\begin{equation*}
\pi(x) > 0 \qquad \forall \: x \in E.
\end{equation*}
Da nämlich $\pi \neq 0$, gibt es ein $z \in E$ mit $\pi (z) > 0$. Aus der Irreduzibilität von $P$ folgt weiterhin, dass zu jedem $x \in E \setminus \lbrace z \rbrace$ ein $n \in \mathbb{N}$ existiert mit $p_{n}(z,x)>0$. Also,
\begin{equation*}
\pi (x) = (\pi P^{n})(x) = \sum_{y \in E} \pi (y) p_{n} (y,x) \geq
\underbrace{\pi (z)}_{>0} \underbrace{ p_{n} (z,x)}_{>0} > 0.
\end{equation*} 
\item[d)] Ist $(X_{n})_{n \in \mathbb{N}_{0}}$ eine Markovkette mit Zustandsraum E und Übergangsmatrix $P$. Wenn $\pi$ eine Gleichgewichtsverteilung ist, so gilt für jedes $n \in \mathbb{N}_{0}$
\begin{equation*}
\mathbb{P}_{\pi}[X_{n} = x] = \sum_{y \in E} \pi (y) \mathbb{P}_{y}[X_{n} = x] = \sum_{y \in E} \pi (y) p_{n}(y,x) = \pi (x).  
\end{equation*}
Insbesondere ist
\begin{equation*}
\mathbb{P}_{\pi} [X_{k+1} = x_{1},...,X_{k+n} = x_{n}] = \sum_{y \in E} \mathbb{P}_{\pi} [X_{k} = y] \mathbb{P}_{\pi} [X_{k+1} = x_{1},...,X_{k+n} = x_{n} \: | \: X_{k} = y]
\end{equation*}
\begin{equation*}
\stackrel{\mathrm{Satz} \: \ref{vorangegangene und zukünftige Ereignisse}}{=} \sum_{y \in E} \pi (y)  \mathbb{P}_{y} [X_{1} = x_{1},...,X_{n} = x_{n}]
\end{equation*}
\begin{equation*}
\mathbb{P}_{\pi} [X_{1} = x_{1},...,X_{n} = x_{n}]
\end{equation*}
\item[e)] Für $\pi_{1}, \pi_{2} \in Inv(P)$ und $\lambda \in [0,1]$ gilt $(\lambda \pi_{1} + (1- \lambda) \pi_{2})[E] = \lambda + (1-\lambda)=1$ und
\begin{equation*}
(\lambda \pi_{1} + (1- \lambda) \pi_{2})P = \lambda \pi_{1}P + (1- \lambda) \pi_{2}P = \lambda \pi_{1} + (1- \lambda) \pi_{2}.
\end{equation*}
Folglich ist die Menge $Inv(P)$ der Gleichgewichtsverteilungen konvex.
\end{itemize}