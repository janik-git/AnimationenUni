\label{doppelt stochastische Übergangsmatrizen}
Sei $(X_{n})_{n \in \mathbb{N}_{0}}$ eine $(\nu,P)$-Markovkette mit Zustandsraum E, wobei die Übergangsmatrix $P$ folgende Eigenschaft besitzt
\begin{equation*}
\sum_{y \in E} p(y,x) = 1 \qquad \forall \: x \in E,
\end{equation*}
d.h. P ist doppelt stochastisch. Ein Spezialfall von doppelt stochastischen Matrizen ist
\begin{equation*}
 p(x,y) = \mu(y-x) \qquad, \quad x,y \in E
\end{equation*} 
für ein Wahrscheinlichkeitsmaß $\mu$ auf E. Dann ist $\pi(x) = 1$ für alle $x \in E$ ein invariantes Maß, denn 
\begin{equation*}
\sum_{y \in E} \pi(y) p(y,x) \sum_{y \in E} p(y,x) = 1 = \pi(x) \qquad , \quad x \in E.
\end{equation*}\label{doppelt stochastische Übergangsmatrizen}
Sei $(X_{n})_{n \in \mathbb{N}_{0}}$ eine $(\nu,P)$-Markovkette mit Zustandsraum E, wobei die Übergangsmatrix $P$ folgende Eigenschaft besitzt
\begin{equation*}
\sum_{y \in E} p(y,x) = 1 \qquad \forall \: x \in E,
\end{equation*}
d.h. P ist doppelt stochastisch. Ein Spezialfall von doppelt stochastischen Matrizen ist
\begin{equation*}
 p(x,y) = \mu(y-x) \qquad, \quad x,y \in E
\end{equation*} 
für ein Wahrscheinlichkeitsmaß $\mu$ auf E. Dann ist $\pi(x) = 1$ für alle $x \in E$ ein invariantes Maß, denn 
\begin{equation*}
\sum_{y \in E} \pi(y) p(y,x) \sum_{y \in E} p(y,x) = 1 = \pi(x) \qquad , \quad x \in E.
\end{equation*}

\textbf{Beispiel 3.15}[Einfache, asymmetrische Irrfahrt auf $\mathbb{Z}$]
\label{Einfache, asymmetrische Irrfahrt auf Z}
Sei $(X_{n})_{n \in \mathbb{N}_{0}}$ eine Markovkette auf $E=\mathbb{Z}$ mit $p(x,x+1) = p$ und $p(x,x-1)=1-p$ für alle $x \in E$ und $p \in (0,1)$. Nach Beispiel $\ref{doppelt stochastische Übergangsmatrizen}$ ist $\pi(x) = 1$ für alle $x \in E$ ein invariantes Maß. Für  $p \neq \dfrac{1}{2}$ ist zudem $\pi(x) = {\left( \dfrac{p}{1-p} \right)}^{x}$, $x \in E$ ein invariantes Maß, denn
\begin{equation*}
\sum_{y \in E} \pi(y) p(y,x) = \pi(x-1)p(x-1,x) + \pi(x+1)p(x+1,x)= {\left( \dfrac{p}{1-p} \right)}^{x}(1-p+p) = \pi(x)
\end{equation*}