$"\Rightarrow"$ Da $\pi$ reversibel bzgl. $P$ ist, ist folglich $\pi$ ein invariantes Maß. Da $P$ zudem irreduzibel ist, so folgt aus Bemerkung $\ref{Auflistende Bemerkung zu invarianten Maßen}$ c), dass $\pi(x) > 0$ für alle $x \in E$
\\
\\
\dashuline{zu zeigen}: $p(x,y)>0 \quad \Rightarrow \quad p(y,x)>0$
\\
\\
Sei also $p(x,y)>0$. Dann ergibt sich aus der "detailed balance"   $\:$Bedingung
\begin{equation*}
p(y,x) = \dfrac{\pi(y)}{\pi(y)} p(y,x) = \underbrace{\dfrac{\pi(x)}{\pi(y)}}_{>0} \underbrace{p(x,y)}_{>0} > 0.
\end{equation*}
Betrachte nun $x_{0},...,x_{n} \in E$ mit $x_{n} = x_{0}$ und  $\prod_{i=1}^{n} p(x_{i},x_{i-1})>0$ .
\\
\\
\dashuline{zu zeigen}: $\prod_{i=1}^{n} \dfrac{p(x_{i-1},x_{i})}{p(x_{i},x_{i-1})} = 1$
\\
\\
Wiederum ergibt sich aus der "detailed balance" Bedingung
\begin{equation*}
\prod_{i=1}^{n} \dfrac{p(x_{i-1},x_{i})}{p(x_{i},x_{i-1})} = \prod_{i=1}^{n} \dfrac{\pi(x_{i-1})}{\pi(x_{i-1})} \dfrac{p(x_{i-1},x_{i})}{p(x_{i},x_{i-1})} = \prod_{i=1}^{n} \dfrac{\pi(x_{i})}{\pi(x_{i-1})} = \dfrac{\pi(x_{n})}{\pi(x_{0})} \stackrel{x_{0} = x_{n}}{=} 1
\end{equation*}
$"\Leftarrow"$ Für ein festes $z \in E$ setze $\pi(z) = 1$. Aus der Irreduzibilität von $P$ folgt, dass zu jedem $x \in E$ ein $n \in \mathbb{N}$ existiert mit $p_{n}(z,x) > 0$. Fo0lglich existieren $x_{0},...,x_{n} \in E$ mit
\begin{equation*}
x_{0} = z \quad, \: x_{n} = x \quad und  \quad \prod_{i=1}^{n} p(x_{i},x_{i-1})>0.
\end{equation*}
Definiere
\begin{equation*}
\pi(x) = \prod_{i=1}^{n} \dfrac{p(x_{i-1},x_{i})}{p(x_{i},x_{i-1})}
\end{equation*}
\\
\\
\dashuline{zu zeigen}: $\pi(x)$ ist unabhängig vom gewählten Pfad
\\
\\
Sei $(x_{0}',...,x_{m}')$ ein weiterer Pfad in E mit $x_{0}' = z, \: x_{m}' = x$ und $\prod_{j=1}^{m} p(x_{j}',x_{j-1}')>0$. Dann folgt aus (i), dass auch $\prod_{j=1}^{m} p(x_{j-1}',x_{j}')>0$. Setze
\begin{equation*}
(y_{0},...,y_{n+m}) = (x_{0},...,x_{n},x_{m-1}',...,x_{0}')
\end{equation*}
Dann gilt
\begin{equation*}
y_{0} = y_{n+m} = z \quad und \quad \prod_{i=1}^{n+m} p(y_{i},y_{i-1}) \stackrel{x_{n}'=x_{n}}{=} \prod_{i=1}^{n} p(x_{i},x_{i-1})  \prod_{j=1}^{m} p(x_{j-1}',x_{j}')>0
\end{equation*}
Also folgt aus (ii)
\begin{equation*}
\pi(x) = \prod_{i=1}^{n} \dfrac{p(x_{i-1},x_{i})}{p(x_{i},x_{i-1})} = \prod_{j=1}^{m} \dfrac{p(x_{j-1}',x_{j}')}{p(x_{j}',x_{j-1}')} \cdot \underbrace{\prod_{i=1}^{n+m} \dfrac{p(y_{i-1},y_{i})}{p(y_{i},y_{i-1})}} _{=1} = \prod_{j=1}^{m} \dfrac{p(x_{j-1}',x_{j}')}{p(x_{j}',x_{j-1}')}
\end{equation*}
Folglich ist $\pi(x)$ unabhängig vom gewählten Pfad.
\\
\\
\dashuline{zu zeigen}: $\pi(x)p(x,y) = \pi(y) p(y,x)$
\\
\\
Falls $p(x,y)=0$, so ist aufgrund von (i) auch $p(y,x)=0$ und die Aussage ist trivial. sei nun also $p(x,y)>0$. Dann ist wegen (i) auch $p(y,x)>0$. Zudem gilt
\begin{equation*}
\pi(x)p(x,y) = \prod_{i=1}^{n} \dfrac{p(x_{i-1},x_{i})}{p(x_{i},x_{i-1})}  \cdot p(x,y)
\end{equation*}
\begin{equation*}
= \prod_{i=1}^{n} \dfrac{p(x_{i-1},x_{i})}{p(x_{i},x_{i-1})}  \cdot \dfrac{p(x,y)}{p(y,x)} \cdot p(y,x) = \pi(y)p(y,x),
\end{equation*}
da mit $x_{n+1}:= y$ der Pfad $(x_{0},...,x_{n},x_{n+1})$ die Eigenschaft hat, dass $x_{0} = z, \: x_{n+1} = y$ und $\prod_{i=1}^{n+1} p(x_{i},x_{i-1})>0$.