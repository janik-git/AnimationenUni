Betrachte eine Markovkette $(X_{n})_{n \in \mathbb{N}_{0}}$ mit Zustandsraum $E = \lbrace 0,1,2,3 \rbrace$, dessen Übergangsmatrix $P$ durch den folgenden Übergangsgraphen beschrieben wird
\begin{figure}[H].
\centering
\includegraphics[scale=0.55]{Beispiel 33}
\caption{Übergangsgraph auf $\lbrace 0,1,2,3 \rbrace$}
\end{figure}
\noindent
Dann bilden die Zustände $\lbrace 0 \rbrace$ und $\lbrace 2,3 \rbrace$ jeweils eine kommunizierende, positiv rekurrente Klasse, während der Zustand $\lbrace 1 \rbrace$ transient ist. Folglich sind
\begin{equation*}
\pi_{1} = \mathbbm{1}_{\lbrace 0 \rbrace} \quad und \quad \pi_{2} = \dfrac{2}{3}\mathbbm{1}_{\lbrace 2 \rbrace} + \dfrac{1}{3}\mathbbm{1}_{\lbrace 3 \rbrace}
\end{equation*}
die beiden Gleichgewichtsverteilungen in $Inv(p)$. Zudem ist $\mathbb{E}_{2}[S_{\lbrace 2 \rbrace}] = \dfrac{3}{2}$.