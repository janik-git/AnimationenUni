(Existenz) Dies folgt direkt aus Satz $\ref{aufzählungen existenz von invarianten Maßen}$ a) und b).
\\
(Eindeutigkeit bis auf konstantes Vielfaches) Sei also $\pi$ ein invariantes Maß bzgl. $P$ mit $\pi(y)=0$ für alle $y \in K^{C}$. Falls $\pi(y) = \infty$ bzw. $\pi(y)=0$ für alle $y \in K$, so gilt für ein festes $x \in K$, dass
\begin{equation*}
\pi = c \cdot \mu_{x}
\end{equation*}
für ein festes $x \in K$ und $c \in \lbrace 0, \infty \rbrace$
\\
Sei nun also $\pi(y)$ weder konstant Null bzw. unendlich. Dann existiert ein $x \in K$ mit $\pi(x) \in (0,\infty)$ Betrachte das Maß $\lambda(y):=\pi(y) / \pi(x)$. Dann genügt $\lambda $ den Voraussetzungen von Satz $\ref{aufzählungen existenz von invarianten Maßen}$ a). Folglich ist 
\begin{equation*}
\lambda = \mu_{x} \qquad \Leftrightarrow \qquad \pi = \pi(x)\mu_{x}.
\end{equation*}