Sei $(X_{n})_{n \in \mathbb{N}_{0}}$ eine $(\nu,P)$-Markovkette auf $E = \lbrace 1,2,3,4 \rbrace$, deren stochastische Matrix durch folgenden Übergangsgraphen beschrieben wird
\begin{figure}[H].
\centering
\includegraphics[scale=0.35]{beispiel22}
\caption{Übergangsgraph des stochastischen Prozesses}
\end{figure}
\noindent
Im folgenden soll die Eintrittswahrscheinlichkeit in den Zustand $\lbrace 4 \rbrace$, d.h.
\begin{equation*}
h_{\lbrace    4 \rbrace}(x) := \mathbb{P}_{x}[T_{\lbrace 4 \rbrace} < \infty], \: x \in E
\end{equation*}
bestimmt werden. Nach Satz \ref{Dirichletsatz} genügt es dazu, die minimale Lösung des Dirichletproblems mit $A=\lbrace 4 \rbrace$ zu berechnen. Betrachte somit folgendes Gleichungssystem.
\begin{align*}
     I. \quad h_{\lbrace 4 \rbrace}(1) &= c          & II. \quad h_{\lbrace 4 \rbrace}(2) &= \dfrac{1}{2}(h_{\lbrace 4 \rbrace}(1) + h_{\lbrace 4 \rbrace}(3))  \,  \\ 
  III. \quad h_{\lbrace 4 \rbrace}(3) &= \dfrac{1}{2}(h_{\lbrace 4 \rbrace}(2) + h_{\lbrace 4 \rbrace}(4))   & IV. \quad h_{\lbrace    4 \rbrace}(1) &= 1.
\end{align*}
\begin{center}
$\Leftrightarrow$ $h_{\lbrace 4 \rbrace} = (c, \dfrac{2}{3}c + \dfrac{2}{3}, \dfrac{1}{3}c + \dfrac{5}{6},1)^{T}$
\end{center}
Folglich besitzt obiges Gleichungssystem erst durch die Minimalitätsbedingung eine eindeutige Lösung. Wähle hierzu $c=0$. Dann folgt
\begin{center}
$h_{\lbrace 4 \rbrace}(2) = \mathbb{P}_{2}[T_{\lbrace 4 \rbrace} < \infty] = \dfrac{2}{3}$ und $h_{\lbrace 4 \rbrace}(3) = \mathbb{P}_{3}[T_{\lbrace 4 \rbrace} < \infty] = \dfrac{5}{6}$
\end{center}