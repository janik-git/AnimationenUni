Da nach Aufgabe 2 die Menge $\mathcal{Z}$ der endlich-dimensionalen Rechtecke einen $\cap$-stabilen Erzeuger von $\varepsilon^{ \otimes I}$ bilden und
\begin{equation*}
\mathbb{Q}[\lbrace x \in E^{I} : x_{j} \in A_{j}, \: \forall j \in J\rbrace] = \mathbb{Q}[{\pi_{J}}^{-1}(\times_{j \in J} \: A_{j})] = (\mathbb{Q} \circ {\pi_{J}}^{-1})[\times_{j \in J} \: A_{j}] = \mathbb{Q}_{J}[\times_{j \in J} \: A_{j}]  
\end{equation*}
für alle $\emptyset \neq J \subseteq I$ endlich und $A_{j} \in \varepsilon$ für alle $j \in J$, folgt die Aussage aus dem Eindeutigkeitssatz für Maße. 