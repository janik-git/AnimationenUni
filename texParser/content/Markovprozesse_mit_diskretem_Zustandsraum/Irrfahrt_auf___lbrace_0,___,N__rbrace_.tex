Eine $(\nu,P)$-Markovkette $(X_{n})_{n \in \mathbb{N}_{0}}$ ist eine einfache Irrfahrt auf $E = \lbrace0,...,N \rbrace$ mit Startverteilung $\nu$, wenn $P$ durch folgenden Übergangsgraphen gegeben ist
\begin{figure}[H].
\centering
\includegraphics[scale=0.36]{beispiel19}
\caption{Irrfahrt auf $\lbrace 0,...,N \rbrace$}
\end{figure}
\noindent
Der Rand x = 0 heißt absorbierend, wenn p(0,0) = 1 (d.h. a=0) bzw. refklektierend, wenn p(0,1) = 1 (d.h. a=1).
Eine $(\nu,P)$-Markovkette $(X_{n})_{n \in \mathbb{N}_{0}}$ ist eine einfache Irrfahrt auf $E = \lbrace0,...,N \rbrace$ mit Startverteilung $\nu$, wenn $P$ durch folgenden Übergangsgraphen gegeben ist
\begin{figure}[H].
\centering
\includegraphics[scale=0.36]{beispiel19}
\caption{Irrfahrt auf $\lbrace 0,...,N \rbrace$}
\end{figure}
\noindent
Der Rand x = 0 heißt absorbierend, wenn p(0,0) = 1 (d.h. a=0) bzw. refklektierend, wenn p(0,1) = 1 (d.h. a=1).



\textbf{Beispiel 1.7}[Irrfahrt auf dem Torus ${(\mathbb{Z}/N\mathbb{Z})}^{d}$]
Sei $E = (\mathbb{Z} \: mod \: N)$ = $( \mathbb{Z}/N)$ für $N \in \mathbb{N}$, $N \ge 2$, $\mu$ eine Wahrscheinlichkeitsverteilung auf E  und $P =(p(x,y))_{x,y \in E}$ mit $p(x,y) = \mu(y-x)$. Dann ist die $(\nu,P)$-Markovkette $(X_{n})_{n \in \mathbb{N}_{0}}$ eine Irrfahrt auf dem Torus mit Startverteilung $\nu$, wenn $P$ durch folgende Übergangsmatrix gegeben ist
\begin{figure}[H].
\centering
\includegraphics[scale=0.3]{beispiel110}
\caption{Irrfahrt auf dem Torus}
\end{figure}