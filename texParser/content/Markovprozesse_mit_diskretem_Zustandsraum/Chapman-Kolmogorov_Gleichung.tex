\label{Chapman-Kolmogorov Gleichung}
Für jede stochastische Matrix $P =(p(x,y))_{x,y \in E}$  sei $P^{n} =(p_{n}(x,y))_{x,y \in E}$, $n \in \mathbb{N}_{0}$. Dann gilt für alle $m,n \in \mathbb{N}_{0}$
\begin{equation*}
p_{m+n} (x,y) = \sum_{z \in E} \: p_{m}(x,z) p_{n}(z,y) \qquad \forall \: x,y \in E
\end{equation*}
\begin{equation*}
\bbordermatrix{
  & A   & B   & C  \cr
A & 0 & 0.5 & 0.5 \cr
B & 1 & 0 & 0 \cr
C & 0.5 & 0.5 & 0  \cr
}
\end{equation*}
\newline
\newline
Man kann leicht nachprüfen, dass $P$ den Anforderungen in Definition 1.8 genügt. Im weiteren seien wir an $p_{2}(A,A)$ interessiert. Also der Wahrscheinlichkeit, nach zwei Durchläufen wieder im Startpunkt A zu sein. Chapman-Kolmogorov liefert uns:  
\begin{equation*}
p_{2} (A,A) = \sum_{z \in E} \: p_{1}(A,z) p_{1}(z,A) 
\end{equation*}
\begin{equation*}
= p(A,A) \: \cdot \: p(A,A) \: + \: p(A,B) \: \cdot \: p(B,A) \: + \: p(A,C) \: \cdot \: p(C,A) = 0.75
\end{equation*}
Wobei hier alle Möglichkeiten von A nach A in zwei Durchläufen zu gelangen betrachtet wurden. Ist man ferner an der Matrix $P^{2}$ interessiert, so ist es demnach hinreichend $P = P \cdot P$ zu berechnen
\begin{table}[H]
  \centering
  \begin{tabular}{c|l}
       &  $ \begin{bmatrix} 0 & 0.5 & 0.5 \\ 1 & 0 & 0 \\ 0.5 & 0.5 & 0 \end{bmatrix} $ \\[7mm]
      \hline \\ [-3mm]
    $ \begin{bmatrix} 0 & 0.5 & 0.5 \\ 1 & 0 & 0 \\ 0.5 & 0.5 & 0 \end{bmatrix} $ &
    $ \begin{bmatrix} \textcolor{red} {0 \cdot 0 + 0.5 \cdot 1 + 0.5 \cdot 0.5 }& 0.25 & 0\\ 
                 0 & 0.5 & 0.5 \\ 
                  0.5 & 0.25 & 0.25    \end{bmatrix} $
  \end{tabular}
\end{table}