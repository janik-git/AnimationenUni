\dashuline{zu zeigen}: $h_{A}$ ist eine Lösung des obigen Dirichletproblems.
\\
\\
Da $\lbrace X_{0} \in A \rbrace = \lbrace T_{A} = 0 \rbrace$, ist folglich $h_{A}(x) = 1$ für alle $x \in A$.
\\
Sei also nun $x \in A^{C}$. Dann folgt mit Satz \ref{vorangegangene und zukünftige Ereignisse}
\begin{equation*}
\mathbb{P}_{x}[T_{A} < \infty] = \mathbb{P}_{x}[T_{A} \in \mathbb{N}] = \sum_{y \in E} \mathbb{P}_{x}[T_{A} \in \mathbb{N}, X_{1} = y]
\end{equation*}
\begin{equation*}
= \sum_{y \in E} \mathbb{P}[X_{1} = y \: | \: X_{0} = x] \cdot \mathbb{P}_{\nu}[T_{A} \in \mathbb{N} \: | \: X_{0} = x, X_{1} = y]
\end{equation*}
\begin{equation*}
\stackrel{\mathrm{Satz} \: \ref{vorangegangene und zukünftige Ereignisse}}{=} \sum_{y \in E} p(x,y) \cdot \mathbb{P}_{y}[T_{A} \in \mathbb{N}_{0}]
\end{equation*}
Also,
\begin{equation*}
h_{A}(x) = \mathbb{P}_{x}[T_{A} < \infty] = \sum_{y \in E} p(x,y) \cdot \mathbb{P}_{y}[T_{A} \in \mathbb{N}_{0}] = \sum_{y \in E} p(x,y) \cdot h_{A}(y) \Leftrightarrow (Lh_{A})(x) = 0
\end{equation*}
\dashuline{zu zeigen}: $h_{A}$ ist die kleinste, nichtnegative Lösung des folgenden Dirichletproblems
\begin{equation*}
(Lh)(x) = 0, \: x \notin A
\end{equation*}
\begin{equation*}
h(x) = 1, \: x \in A
\end{equation*}
\dashuline{zu zeigen}: Für alle $N \in \mathbb{N}_{0}$ gilt $\mathbb{P}_{x}[T_{A} \leq N] \leq h(x)$, $x \in E$
\\
\\
Für jedes $N \in \mathbb{N}_{0}$ und $x \in A$ folgt $P_{x}[T_{A} \leq N] = 1 = h(x)$. Für jedes $x \in A^{C}$ ergibt sich mittels vollständiger Induktion über N
\\
\\
\textbf{IA} $N=0$: $\mathbb{P}_{x}[T_{A} = 0] \stackrel{x \notin A}{=}0 \leq h(x)$
\\
\\
\textbf{IS} $N \to N+1$: Sei also $x \notin A$. Dann gilt mittels Satz \ref{vorangegangene und zukünftige Ereignisse}
\begin{equation*}
P_{x}[T_{A} \leq N+1] = P_{x}[1 \leq T_{A} \leq N+1]
\end{equation*}
\begin{equation*}
= \sum_{y \in E} \mathbb{P}[X_{1} = y \: | \: X_{0} = x] \cdot \mathbb{P}_{\nu}[1 \leq T_{A} \leq N+1 \: | \: X_{0} = x, X_{1} = y]
\end{equation*}
\begin{equation*}
\stackrel{\mathrm{Satz} \: \ref{vorangegangene und zukünftige Ereignisse}}{=} \sum_{y \in E} p(x,y) \cdot \mathbb{P}_{y}[T_{A} \leq N]
\end{equation*}
\begin{equation*}
\stackrel{\mathrm{IV}}{\leq}\sum_{y \in E} p(x,y) \cdot h(y)
\end{equation*}
\begin{equation*}
\stackrel{(Lh)(x) = 0}{=}h(x)
\end{equation*}
Da $\lbrace T_{A} \leq N \rbrace \uparrow \bigcup_{k \in \mathbb{N}} \lbrace T_{A} \leq k \rbrace = \lbrace T_{A} < \infty \rbrace$ für $N \to \infty$, so folgt aus der Stetigkeit des Wahrscheinlichkeitsmaßes $\mathbb{P}_{x}$
\begin{equation*}
h_{A}(x) = \mathbb{P}_{x}[T_{A} < \infty] = \lim_{N \to \infty} \mathbb{P}_{x}[T_{A} \leq N] \leq h(x), \: x \in E
\end{equation*}