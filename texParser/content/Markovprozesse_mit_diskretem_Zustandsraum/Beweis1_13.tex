\underline{Schritt 1} Sei $E = \mathbb{N}$. Setze
\\
\begin{equation}
\label{eq: zwei}
\alpha(0) := 0, \alpha(i) := \sum_{k=1}^{i} \nu[\lbrace k \rbrace] \qquad und \qquad \beta(i,0) := 0, \beta(i,j) := \sum_{k=1}^{j} p(i,k) , i \in \mathbb{N}
\end{equation}
Weiterhin definiere die Funktionen $f:[0,1] \to \mathbb{N}$ und $F:\mathbb{N} \times [0,1] \to \mathbb{N}$ durch
\begin{equation*}
f(u) := \sum_{i=1}^{\infty} i \cdot \mathbbm{1}_{\alpha(i-1)\leq u \leq \alpha(i) }
\end{equation*}
\begin{equation*}
F(i,u) := \sum_{j=1}^{\infty} j \cdot \mathbbm{1}_{\beta(i,j-1)\leq u \leq \beta(i,j) }
\end{equation*}
Dann gilt für die durch ($\ref{eq: zwei}$) definierte Folge und alle $i_{0},...,i_{n} \in \mathbb{N}$
\begin{equation*}
\mathbb{P}[X_{0} = i_{0},...,X_{n} = i_{n}]
\end{equation*}
\begin{equation*}
= \mathbb{P}[U_{0} \in [\alpha(i_{0} - 1),\alpha(i_{0})], U_{k} \in [\beta(i_{k-1},i_{k} -1),\beta(i_{k-1},i_{k})] \: \: \forall k  \in \lbrace 1,...,n \rbrace ]
\end{equation*}
\begin{equation*}
\stackrel{}{=} \mathbb{P}[U_{0} \in [\alpha(i_{0} - 1),\alpha(i_{0})] \cdot \prod_{k=1}^{n} \mathbb{P}[ U_{k} \in [\beta(i_{k-1},i_{k} -1),\beta(i_{k-1},i_{k})]]
\end{equation*}
\begin{equation*}
= \nu(i_{0})p(i_{0},i_{1}) \cdot .... \cdot p(i_{n-1},i_{n})   
\end{equation*}
Also ist $(X_{n})_{n \in \mathbb{N}_{0}}$ nach Satz $\ref{Besitzen Markovketten die Markoveigenschaft}$ (i) eine $(\nu,P)$-Markovkette.
\\
\\
\underline{Schritt 2} Da E abzählbar ist, gibt es eine bijektive Funktion $\varphi: \mathbb{N} \to E$. Die Aussage des Satzes folgt somit aus Schritt 1. 