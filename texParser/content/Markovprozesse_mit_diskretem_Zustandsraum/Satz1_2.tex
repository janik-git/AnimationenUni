\label{"Satz 1.8"}
Sei ${(X_{n})}_{n \in \mathbb{N}_{0}}$ eine $(\nu,P)$-Markovkette mit Zustandsraum E. Dann gilt
\begin{equation*}
\mathbb{P}[X_{n} = x] = \sum_{x_{0},..,x_{n-1} \in E} \nu(x_{0}) p(x_{0},x_{1}) \cdot ...\cdot p(x_{n-1},x) = (\nu P^{n})(x) \qquad \forall n \in \mathbb{N}_{0}, \: x \in E
\end{equation*}
Insbesondere gilt für alle $m,n \in \mathbb{N}_{0}$ und alle $x,y \in E$ mit $\mathbb{P}[X_{m} = x]>0$
\begin{equation*}
\mathbb{P}[X_{m+n}=y \: | \: X_{m} = x] = (P^{n})(x,y) = p_{n}(x,y).
\end{equation*}