Sei G = (V,E(V)) ein Graph mit Knotenmenge V und Kantenmenge E(V).
\\
\\
\underline{Schreibweise}: $x \sim y$ $:\Leftrightarrow$ $x,y \: \in V \:$ sind durch eine Kante aus E(V) verbunden. Betrachte
\begin{equation*}
p(x,y)=
\begin{cases}
\dfrac{1}{deg(x)} &  ,x \sim y\\
0 & ,\mathrm{sonst}
\end{cases}
\end{equation*}
wobei deg(x) die Anzahl der von x ausgehenden Kanten ist. Dann ist $P =(p(x,y))_{x,y \in V}$ eine stochastische Matrix, und die ($\nu$,P)-Markovkette  $(X_{n})_{n \in \mathbb{N}_{0}}$ bezeichnet man als Irrfahrt auf dem Graphen G mit Startverteilung $\nu$. 
\begin{figure}[H].
\centering
\includegraphics[scale=0.35]{beispiel111}
\caption{Irrfahrt auf einem Graphen}
\end{figure}
\noindent
Dabei ist $V = \lbrace 1,2,3,4 \rbrace$ die Knotenmenge und $E(V) = \lbrace \lbrace 1,2 \rbrace, \lbrace 1,3 \rbrace, \lbrace 2,3 \rbrace, \lbrace 3,4 \rbrace \rbrace$ die Kantenmenge.