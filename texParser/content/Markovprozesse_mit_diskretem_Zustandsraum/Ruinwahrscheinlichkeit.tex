Betrachte eine einfache asymmetrische Irrfahrt auf $E = \mathbb{N}_{0}$ mit Absorbtion im Zustand 0.
\begin{figure}[H].
\centering
\includegraphics[scale=0.35]{beispiel23}
\caption{asymmetrische Irrfahrt auf $\mathbb{N}_{0}$}
\end{figure}
\noindent
wiederum soll die Eintrittswahrscheinlichkeit in $\lbrace 0 \rbrace$ (= Absorbtionswahrscheinlichkeit) bestimmt werden. Aus Satz \ref{Dirichletsatz} folgt, dass $h_{\lbrace 0 \rbrace}(x) := \mathbb{P}_{x}[T_{\lbrace 0 \rbrace} < \infty], \: x \in E$ die minimale, nichtnegative Lösung des folgenden Dirichletproblems ist:
\begin{equation*}
(Lh_{\lbrace 0 \rbrace})(x) = p(h_{\lbrace 0 \rbrace}(x+1) - h_{\lbrace 0 \rbrace}(x)) + q(h_{\lbrace 0 \rbrace}(x-1) - h_{\lbrace 0 \rbrace}(x)) = 0, \: x \neq 0
\end{equation*}
\begin{equation*}
h_{\lbrace 0 \rbrace}(0) = 1
\end{equation*}
\underline{Beh.}: Für $p \neq q$ ist die Lösung von $(Lh)(x) = 0$ für alle $x \in E$ gegeben durch
\begin{equation*}
h(x) = a + b \cdot (\dfrac{q}{p})^{x}
\end{equation*}
Es gilt nämlich für $x \in \mathbb{N}$
\begin{equation*}
(Lh)(x) = pb (\dfrac{q}{p})^{x}(\dfrac{q}{p} - 1) + qb (\dfrac{q}{p})^{x} (\dfrac{p}{q} - 1) = b (\dfrac{q}{p})^{x}(q-p + p-q) = 0
\end{equation*}
\dashuline{Fall 1}: p<q
\\
\\
Da $h_{\lbrace 0 \rbrace}(x) \in [0,1]$ für alle $x \in \mathbb{N}_{0}$, folgt $b=0$ und, wegen $h_{\lbrace 0 \rbrace}(0) = 1, \: a=1$. Also
\begin{equation*}
h_{\lbrace 0 \rbrace}(x) = \mathbb{P}_{x}[T_{\lbrace 0 \rbrace} < \infty] = 1 \qquad \forall x \in \mathbb{N}_{0}
\end{equation*}
\dashuline{Fall 2}: q<p
\\
\\
Aus $h_{\lbrace 0 \rbrace}(0) = 1$ folgt zunächst einmal, dass $b = 1-a$. Also
\begin{equation*}
[0,1] \ni h_{\lbrace 0 \rbrace}(x) = (\dfrac{q}{p})^{x} + a(1- (\dfrac{q}{p})^{x}) \qquad \forall x \in \mathbb{N}_{0} \: \Rightarrow a \geq 0
\end{equation*}
Somit impliziert erst die Minimalitätsbedingung, dass $a=0$ ist, d.h.
\begin{equation*}
h_{\lbrace 0 \rbrace}(x) = \mathbb{P}_{x}[T_{\lbrace 0 \rbrace} < \infty] = (\dfrac{q}{p})^{x}
\end{equation*}
\underline{Beh.}: Für $p=q$ ist die Lösung von $(Lh)(x) = 0$ für alle  $x \in E$ gegeben durch 
\begin{equation*}
h(x) = a + bx
\end{equation*}
Denn für alle $x \in \mathbb{N}$ gilt $(Lh)(x) = \dfrac{1}{2}b(x + 1 -x) + \dfrac{1}{2}b(x -1 -x) = 0$
\\
Da $h_{\lbrace 0 \rbrace}(x) \in [0,1]$ für alle $x \in \mathbb{N}_{0}$, so folgt $b=0$. Wegen $h_{\lbrace 0 \rbrace}(0) = 1$ ist zudem $a = 1$:
\begin{equation*}
h_{\lbrace 0 \rbrace}(x) = \mathbb{P}_{x}[T_{\lbrace 0 \rbrace} \leq \infty] = 1
\end{equation*}