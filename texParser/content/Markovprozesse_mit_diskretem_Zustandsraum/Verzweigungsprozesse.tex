Sei $X_{n}$ die Anzahl der Individuen in der n-ten Generation. Jedes Individuum der n-ten Generation wird unabhängig von allen anderen in der folgenden Generation mit Wahrscheinlichkeit $\mu(y)$ mit $y \in \mathbb{N}_{0}$ Nachkommen ersetzt. Dann lässt sich $(X_{n})_{n \in \mathbb{N}_{0}}$ durch eine Markovkette auf $E = \mathbb{N}_{0}$ mit Übergangsmatrix $P =(p(x,y))_{x,y \in E}$ 
\begin{equation*}
p(x,y) = \mu^{*x}(y) = \sum_{y_{1} +...+ y_{x} = y} \: \mu(y_{1})... \mu(y_{2}) \qquad \mathrm{mit} \: \mu^{*0}(y) = \mathbbm{1}_{\lbrace 0 \rbrace}(y)
\end{equation*}
\textcolor{red}{Frage?} Woher kommt die Faltung?
\\
\\
Sei $({X_{n}}^{(i)})_{n,i \in \mathbb{N}_{0}}$ eine Folge u.i.v. Zufallsvariablen mit $\mathbb{P}[{X_{n}}^{(i)} = k] = \mu(k)$. Setze
\begin{equation*}
X_{0} = x \qquad  \mathrm{und}  \qquad X_{n} = \sum_{i = 1}^{x_{n-1}} X_{n-1}^{(i)}
\end{equation*}
Dann gilt
\begin{equation*}
\mathbb{P}[X_{n+1} = x_{n+1} \: | \: X_{0} = x_{0},...,X_{n} = x_{n}] = \mathbb{P}[X_{n}^{1}+...+X_{n}^{(x_{n})} = x_{n+1}]
\end{equation*}
\begin{equation*}
= \mu^{*x_{n}}[x_{n+1}] = p(x_{n},x_{n+1}).
\end{equation*}