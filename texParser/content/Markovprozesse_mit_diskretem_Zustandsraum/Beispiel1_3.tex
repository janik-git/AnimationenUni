Sei $(X_{n})_{n \in \mathbb{N}_{0}}$ eine Folge von unabhängigen Zufallsvariablen mit Werten in E. Dann gilt für jedes $n \in \mathbb{N}_{0}$ und $x_{0},...,x_{n+1} \in E$ mit $\mathbb{P}[X_{0} = x_{0},...,X_{n} = x_{n}]>0$
\begin{equation*}
\mathbb{P}[X_{n+1} = x_{n} \: | \: X_{0} = x_{0},...,X_{n} = x_{n}] = \mathbb{P}[X_{n+1} = x_{n+1}] = \mathbb{P}[X_{n+1} = x_{n} \: | \: X_{n} = x_{n}]
\end{equation*}
Folglich besitzt die Folge $(X_{n})_{n \in \mathbb{N}_{0}}$ die Markoveigenschaft. Falls die Zufallsvariablen zudem identisch verteilt sind, d.h. $\mathbb{P}[X_{n+1} = x] = \mathbb{P}[X_{1} = x]$ für alle $n \in \mathbb{N}_{0}$, so hat $(X_{n})_{n \in \mathbb{N}_{0}}$ die zeithomogene Markoveigenschaft.