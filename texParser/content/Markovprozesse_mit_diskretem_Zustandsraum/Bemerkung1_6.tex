Sei $(X_{n})_{n \in \mathbb{N}_{0}}$ ein E-wertiger stochastischer Prozess, der die zeithomogene Markoveigenschaft besitzt. Dann ist $(X_{n})_{n \in \mathbb{N}_{0}}$ eine Markovkette mit Zustandsraum E, Startverteilung $\nu = \mathbb{P} \circ {X_{0}}^{-1}$ und Übergangsmatrix $P =(p(x,y))_{x,y \in E}$ mit
\begin{equation*}
p(x,y) := \mathbb{P}[X_{1} = y \: | \: X_{0} = x]
\end{equation*}