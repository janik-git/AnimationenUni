Ist $(X_{n})_{n \in \mathbb{N}_{0}}$  eine $(\nu,P)$-Markovkette, so gilt für alle $n \in \mathbb{N}_{0}$, $A \in \varepsilon^{ \otimes \mathbb{N}_{0}}$, $B \subseteq E^{n}$ und $x \in E$ mit $\mathbb{P}_{\nu}[X_{n} = x]>0$
\begin{equation*}
\mathbb{P}[(X_{0},X_{1},...,X_{n-1}) \in  B, (X_{n+1}, X_{n+2},...) \in A \: | \: X_{n} = x]
\end{equation*}
\begin{equation*}
= \mathbb{P}[(X_{0},X_{1},...,X_{n-1}) \in  B \: | \: X_{n} = x] \cdot \mathbb{P}[(X_{n+1}, X_{n+2},...) \in A \: | \: X_{n} = x]
\end{equation*}