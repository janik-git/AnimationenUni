Sei $(Y_{n})_{n \in \mathbb{N}_{0}}$ eine Folge u.i.v. Zufallsvariablen mit 
\\
$\mathbb{P}[Y_{1} = 1] = \mathbb{P}[Y_{1} = -1] =\dfrac{1}{2}$. Setze $X_{0} = 1$ und $X_{n} = Y_{n}, n \in \mathbb{N}$. Dann besitzt $(X_{n})_{n \in \mathbb{N}_{0}}$ wegen Beispiel 1.3 die zeithomogene Markoveigenschaft und ist somit wegen Bemerkung 1.6 eine Markovkette auf dem Zustandsraum $E = \lbrace -1,+1  \rbrace$ mit Startverteilung $\nu = \mathbbm{1}_{\lbrace 1 \rbrace}$ und Übergangsmatrix 
\begin{equation*}
P =(p(x,y))_{x,y \in E} \: \: \:  und \: \: \: p(x,y) = \dfrac{1}{2}
\end{equation*}Sei $(Y_{n})_{n \in \mathbb{N}_{0}}$ eine Folge u.i.v. Zufallsvariablen mit 
\\
$\mathbb{P}[Y_{1} = 1] = \mathbb{P}[Y_{1} = -1] =\dfrac{1}{2}$. Setze $X_{0} = 1$ und $X_{n} = Y_{n}, n \in \mathbb{N}$. Dann besitzt $(X_{n})_{n \in \mathbb{N}_{0}}$ wegen Beispiel 1.3 die zeithomogene Markoveigenschaft und ist somit wegen Bemerkung 1.6 eine Markovkette auf dem Zustandsraum $E = \lbrace -1,+1  \rbrace$ mit Startverteilung $\nu = \mathbbm{1}_{\lbrace 1 \rbrace}$ und Übergangsmatrix 
\begin{equation*}
P =(p(x,y))_{x,y \in E} \: \: \:  und \: \: \: p(x,y) = \dfrac{1}{2}
\end{equation*}

%Begin_Zusatz
\textbf{Beispiel 1.7}[Irrfahrt auf $\mathbb{Z}$]
Sei $E = \mathbb{Z}$ mit
\\
\\
$p(x,y)=
\begin{cases}
\dfrac{1}{2} &  |x - y| = 1\\
0 & sonst
\end{cases}$
\\
\\
Diese Markovkette beschreibt ein Teilchen, das pro Zeiteinheit auf $\mathbb{Z}$ um eins nach rechts oder
links springt, und zwar immer mit Wahrscheinlichkeit $\dfrac{1}{2}$. Die Übergangsmatrix $P = (p(x,y))_{x,y \in \mathbb{Z}}$
ist dann eine unendlich große Triagonalmatrix, die auf der Hauptdiagonalen ausschließlich Nullen hat und auf den beiden Nebendiagonalen immer den Wert $\dfrac{1}{2}$.
Das Anfangsstück eines Pfades der symmetrischen Irrfahrt (mit Start in Null) kann zum Beispiel so aussehen (linear interpoliert):
\begin{figure}[H].
\centering
\includegraphics[scale=0.55]{beispiel18}
\caption{Realisierung einer symmetrischen Irrfahrt}
\end{figure}Sei $(Y_{n})_{n \in \mathbb{N}_{0}}$ eine Folge u.i.v. Zufallsvariablen mit 
\\
$\mathbb{P}[Y_{1} = 1] = \mathbb{P}[Y_{1} = -1] =\dfrac{1}{2}$. Setze $X_{0} = 1$ und $X_{n} = Y_{n}, n \in \mathbb{N}$. Dann besitzt $(X_{n})_{n \in \mathbb{N}_{0}}$ wegen Beispiel 1.3 die zeithomogene Markoveigenschaft und ist somit wegen Bemerkung 1.6 eine Markovkette auf dem Zustandsraum $E = \lbrace -1,+1  \rbrace$ mit Startverteilung $\nu = \mathbbm{1}_{\lbrace 1 \rbrace}$ und Übergangsmatrix 
\begin{equation*}
P =(p(x,y))_{x,y \in E} \: \: \:  und \: \: \: p(x,y) = \dfrac{1}{2}
\end{equation*}

%Begin_Zusatz
\textbf{Beispiel 1.7}[Irrfahrt auf $\mathbb{Z}$]
Sei $E = \mathbb{Z}$ mit
\\
\\
$p(x,y)=
\begin{cases}
\dfrac{1}{2} &  |x - y| = 1\\
0 & sonst
\end{cases}$
\\
\\
Diese Markovkette beschreibt ein Teilchen, das pro Zeiteinheit auf $\mathbb{Z}$ um eins nach rechts oder
links springt, und zwar immer mit Wahrscheinlichkeit $\dfrac{1}{2}$. Die Übergangsmatrix $P = (p(x,y))_{x,y \in \mathbb{Z}}$
ist dann eine unendlich große Triagonalmatrix, die auf der Hauptdiagonalen ausschließlich Nullen hat und auf den beiden Nebendiagonalen immer den Wert $\dfrac{1}{2}$.
Das Anfangsstück eines Pfades der symmetrischen Irrfahrt (mit Start in Null) kann zum Beispiel so aussehen (linear interpoliert):
\begin{figure}[H].
\centering
\includegraphics[scale=0.55]{beispiel18}
\caption{Realisierung einer symmetrischen Irrfahrt}
\end{figure}

%End_Zusatz

\textbf{Beispiel 1.7}[Irrfahrt auf $\mathbb{Z}^{d}$]
Sei $\mu$ eine Wahrscheinlichkeitsverteilung auf $\mathbb{Z}^{d}$ . Setze
\begin{equation*}
p(x,y) = \mu(x-y) \qquad \forall \: x,y \in \mathbb{Z}^{d}
\end{equation*}
Offensichtlich ist $P =(p(x,y))_{x,y \in \mathbb{Z}^d}$ eine stochastische Matrix. Dann nennt man $(\mathbbm{1}_{x},\mathbb{P})$-Markovkette $(X_{n})_{n \in \mathbb{N}_{0}}$ eine Irrfahrt(random walk) auf $\mathbb{Z}^{d}$ mit Start in $x \in \mathbb{Z}^{d}$. Im Spezialfall, dass
\begin{equation*}
\mu(x)=
\begin{cases}
\dfrac{1}{2d} &  ,|x|=1\\
0 & ,\mathrm{sonst}
\end{cases}
\end{equation*}
nennt man $(X_{n})_{n \in \mathbb{N}_{0}}$ eine einfache, symmetrische Irrfahrt.
\begin{figure}[H].
\centering
\includegraphics[scale=0.85]{beispiel17(2)}
\caption{Symmetrische Irrfahrt auf $\mathbb{Z}^2$}
\end{figure}