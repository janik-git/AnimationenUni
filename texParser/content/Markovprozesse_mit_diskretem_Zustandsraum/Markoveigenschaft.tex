Ein stochastischer Prozess $(X_{n})_{n \in \mathbb{N}_{0}}$ auf $(\Omega, \mathfrak{F}, \mathbb{P})$ mit Werten in E besitzt die (elementare) Markoveigenschaft, wenn für jedes $n \in \mathbb{N}_{0}$ und alle $x_{0},...,x_{n+1} \in E$ mit $\mathbb{P}[X_{0} = x_{0},...,X_{n} = x_{n}]>0$ gilt
\begin{equation*}
\mathbb{P}[X_{n+1} = x_{n} \: | \: X_{0} = x_{0},...,X_{n} = x_{n}] = \mathbb{P}[X_{n+1} = x_{n} \: | \: X_{n} = x_{n}] 
\end{equation*}
Falls zudem für alle $n \in \mathbb{N}_{0}$ und $x,y \in E$ gilt, dass
\begin{equation*}
\mathbb{P}[X_{n+1} = y \: | \: X_{n} = x] = \mathbb{P}[X_{1} = y \: | \: X_{0} = x]
\end{equation*}
so besitzt $(X_{n})_{n \in \mathbb{N}_{0}}$ die zeithomogene Markoveigenschaft.