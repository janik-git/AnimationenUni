Sei $X=(X_{n})_{n \in \mathbb{N}_{0}}$ ein stochastischer Prozess mit Zustandsraum E und $A \subseteq E$.
\begin{itemize}
\item[(a)] Die Erste Rückkehr-bzw. Treffzeit $S_{A}$ und die Eintrittszeit $T_{A}$ ist gegeben durch 
\begin{equation*}
S_{A}(\omega) := \inf \lbrace n \in \mathbb{N} : X_{n}(\omega) \in A \rbrace
\end{equation*}
\begin{equation*}
T_{A}(\omega) := \inf \lbrace n \in \mathbb{N}_{0} : X_{n}(\omega) \in A \rbrace
\end{equation*}
sind Stoppzeiten, denn
\begin{equation*}
\lbrace S_{A} = n \rbrace = \lbrace X_{1} \notin A,...,X_{n-1} \notin A, X_{n} \in A  \rbrace \in \mathfrak{F}_{n}^{X}
\end{equation*}
\begin{equation*}
\lbrace T_{A} = n \rbrace = \lbrace X_{0} \notin A,...,X_{n-1} \notin A, X_{n} \in A  \rbrace \in \mathfrak{F}_{n}^{X}
\end{equation*}
Insbesondere gilt $\mathbb{P}[S_{A} = T_{A} \: | \: X_{0} = x] = 1$ für alle $x \notin A$
\item[(b)] Die k-te Treffzeit ist gegeben durch 
\begin{equation*}
S_{A}^{0}(\omega) := 0 \: , \: S_{A}^{k}(\omega) := \inf \lbrace n> S_{A}^{k-1}(\omega) \: : \: X_{n} \in A \rbrace \: , \: k \in \mathbb{N}
\end{equation*}
\item[(c)] Jede deterministische Zeit $T(\omega) = t$, $t \in \mathbb{N}_{0}$ ist eine Stoppzeit, da
\begin{equation*}
\lbrace T = n \rbrace \in \lbrace \emptyset, \Omega \rbrace  \in \mathfrak{F}_{n}^{X} \:,\: n \in \mathbb{N}_{0} 
\end{equation*}
\item[(d)] Die letzte Austrittszeit auf der Menge A 
\begin{equation*}
L_{A}(\omega) := \sup \lbrace n \in \mathbb{N}_{0} \: : \: X_{n} \in A \rbrace
\end{equation*}
ist i.A. keine Stoppzeit, da $\lbrace L_{A} = n \rbrace$ davon abhängt, ob $(X_{n+m})_{m \in \mathbb{N}_{0}}$ die Menge A trifft oder nicht.
\end{itemize}