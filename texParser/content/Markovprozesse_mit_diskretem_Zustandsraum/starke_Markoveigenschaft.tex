\label{starke Markoveigenschaft}
Ist $(X_{n})_{n \in \mathbb{N}_{0}}$ eine $(\nu,P)$-Markovkette mit Zustandsraum E und T eine Stoppzeit, so gilt für alle $A \in \varepsilon^{ \otimes \mathbb{N}_{0}}$, $F \in \mathfrak{F}_{T}^{X}$ und $x \in E$ mit $\mathbb{P}_{\nu}[F,X_{T} = x, T < \infty] > 0$
\begin{equation*}
\mathbb{P}_{\nu}[(X_{T},X_{T+1},...) \in A \: | \: F, X_{T} = x, T < \infty] = \mathbb{P}_{x}[(X_{0},X_{1},...) \in A],
\end{equation*}
wobei $X_{T}$ auf $\lbrace T < \infty \rbrace$ definiert ist durch $X_{T}(\omega) := X_{T(\omega)}(\omega)$