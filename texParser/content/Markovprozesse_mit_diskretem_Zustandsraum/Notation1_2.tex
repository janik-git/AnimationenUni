Um die Startverteilung zu betonen, schreiben wir auch $\mathbb{P}_{\nu}$ bzw. $\mathbb{P}_{x}$ falls $\nu = \mathbbm{1}_{ \lbrace x \rbrace}$.
%Begin_Zusatz
Dabei steht $\mathbb{P}_{\nu} = [X_{n} = y]$ bzw. $\mathbb{P}_{x} = [X_{n} = y]$ für die Wahrscheinlichkeit bei Anfangsverteilung $\nu$ bzw. Start im Zustand $x_{0}$ im n-ten Schritt y zu realisieren
%End_Zusatz