\label{zeithomogene Markovkette}
Sei  $P =(p(x,y))_{x,y \in E}$ eine stochastische Matrix und $\nu: E \to [0,1]$ ein Wahrscheinlichkeitsvektor. Ein stochastischer Prozess $(X_{n})_{n \in \mathbb{N}_{0}}$ auf $(\Omega, \mathfrak{F}, \mathbb{P})$ mit Zustandsraum E heißt (zeithomogene) Markovkette mit Übergangsmatrix $P$ und Startverteilung $\nu$ (kurz: $(\nu,P)$-Markovketten), falls
\\
\\
(i) Für alle $n \in \mathbb{N}_{0}$ und $x_{0},...,x_{n+1} \in E$ mit $\mathbb{P}[X_{0} = x_{0},...,X_{n} = x_{n}]>0$ gilt
\begin{equation*}
\mathbb{P}[X_{n+1} = x_{n+1} \: | \: X_{0} = x_{0},...,X_{n} = x_{n}] = p(x_{n},x_{n+1})
\end{equation*}
(ii) Für alle $x_{0} \in E$ gilt
\begin{equation*}
\mathbb{P}[X_{0} = x_{0}]  = \nu(x_{0})
\end{equation*}