\underline{Schritt 1} Für ein beliebiges $k \in \mathbb{N}_{0}$ und $x_{0},...,x_{k} \in E$ betrachte zunächst 
\begin{equation*}
\mathbb{P}_{\nu}[(X_{0},...,X_{n-1}) \in B ,  X_{n} = x, X_{n+i} = x_{i} \: , \forall \: i \in \lbrace 0,1,...,k \rbrace]
\end{equation*}
\begin{equation*}
\stackrel{\mathrm{Satz \: 1.3 \: (i)}}{=} \sum_{(y_{0},...,y_{n-1}) \in B} \nu(y_{0})p(y_{0},y_{1}) \cdot ... \cdot p(y_{n-1},x) \cdot \mathbbm{1}_{ x = x_{0}} \cdot p(x_{0},x_{1})  \cdot ... \cdot p(x_{k-1},x_{k})
\end{equation*}
\begin{equation*}
= \mathbb{P}_{\nu}[(X_{0},...,X_{n-1}) \in B, X_{n} = x] \cdot  \mathbb{P}_{x}[X_{i} = x_{i} \: , \forall \: i \in \lbrace 0,1,...,k \rbrace]
\end{equation*}
Da E diskret ist, folgt somit die Behauptung für alle endlich-dimensionalen Rechteckmengen.
\\
\\
\underline{Schritt 2} Betrachte das Mengensystem
\begin{equation*}
\mathcal{D} = \lbrace A \in \varepsilon^{ \otimes \mathbb{N}_{0}} \: | \: \mathbb{P}_{\nu}[(X_{n},X_{n+1},...) \in A \: | \: (X_{0},...,X_{n-1})  \in B, \: X_{n} = x] = \mathbb{P}_{x}[(X_{0},X_{1},...) \in A] \rbrace
\end{equation*}
Wir wollen zeigen, dass $\mathcal{D}$ ein Dynkinssystem ist.
\\
\\
1. Wir müssen zeigen dass $\mathcal{D}$ Omega(hier $E^{\mathbb{N}_{0}}$) enthält.
\\
\\
Da $\mathbb{P}_{\nu}[(X_{n},X_{n+1},...) \in E^{\mathbb{N}_{0}}  \: | \: (X_{0},...,X_{n-1})  \in B, X_{n} = x] = 1 = \mathbb{P}_{x}[(X_{0},X_{1},...) \in E^{\mathbb{N}_{0}}]$ ist somit $E^{\mathbb{N}_{0}} \in \mathcal{D}$ 
\\
\\
2. Wir müssen Stabiltät unter Komplementbildung zeigen.
\\
\\
Sei $D \in \mathcal{D}$. Dann ist auch $D^{C} \in \mathcal{D}$, denn
\begin{equation*}
\mathbb{P}_{\nu}[(X_{n},X_{n+1},...) \in D^{C} \: | \: (X_{0},...,X_{n-1})  \in B, X_{n} = x]
\end{equation*}
\begin{equation*}
= 1 - \mathbb{P}_{\nu}[(X_{n},X_{n+1},...) \in D \: | \: (X_{0},...,X_{n-1})  \in B, X_{n} = x]
\end{equation*}
\begin{equation*}
\stackrel{D \in \mathcal{D}}{=} 1 - \mathbb{P}_{x}[(X_{0},X_{1},...) \in D]
\end{equation*}
\begin{equation*}
= \mathbb{P}_{x}[(X_{0},X_{1},...) \in D^{C}]
\end{equation*}
\\
\\
3. Wir müssen Abgeschlossenheit unter disjunkter Vereinigung zeigen.
\\
\\
Seien nun $D_{1},D_{2}... \in \mathcal{D}$ disjunkt und $D = \bigcup_{i=1}^{\infty} $ $D_{i}$ Dann gilt
\begin{equation*}
\mathbb{P}_{\nu}[(X_{n},X_{n+1},...) \in D \: | \: (X_{0},...,X_{n-1})  \in B, X_{n} = x]
\end{equation*}
\begin{equation*}
\stackrel{\sigma \mathrm{-Additivität}}{=}  \sum_{i=1}^{\infty} \mathbb{P}_{\nu}[(X_{n},X_{n+1},...) \in D_{i} \: | \: (X_{0},...,X_{n-1})  \in B, X_{n} = x]
\end{equation*}
\begin{equation*}
\stackrel{D_{i} \in \mathcal{D}}{=} \sum_{i=1}^{\infty} \mathbb{P}_{x}[(X_{0},X_{1},...) \in D_{i}]
\end{equation*}
\begin{equation*}
\stackrel{\sigma \mathrm{-Additivität}}{=} \mathbb{P}_{x}[(X_{0},X_{1},...) \in D]
\end{equation*}
Also ist auch $D \in \mathcal{D}$.
\\
\\ 
Da $\mathcal{D}$ nach Schritt 1 auch das $\cap$-stabile Mengensystem $\mathcal{Z}$ der endlich-dimensionalen Rechtecke enthält, folgt aus dem Hauptsatz über Dynkinsysteme  
\begin{equation*}
\mathcal{Z} \subseteq \mathcal{D} \qquad \Rightarrow \qquad \varepsilon^{ \otimes \mathbb{N}_{0}} = \sigma(\mathcal{Z}) = d(\mathcal{Z}) \subseteq \mathcal{D} \subseteq \varepsilon^{ \otimes \mathbb{N}_{0}}
\end{equation*}