$\mathfrak{F}_{T}^{X}$ ist eine $\sigma$-Algebra. Da T eine Stopppzeit bzgl. X ist, ist $\Omega \in \mathfrak{F}_{T}^{X}$. Weiterhin gilt für jedes $n \in \mathbb{N}_{0}$ und $A \in \mathfrak{F}_{T}^{X}$
\begin{equation*}
A^{C} \cap \lbrace T = n \rbrace = \underbrace{(A \cap \lbrace T = n \rbrace)^{C}}_{ \in \mathfrak{F}_{T}^{X}} \cap \underbrace{\lbrace T = n \rbrace}_{ \in \mathfrak{F}_{T}^{X}} \in \mathfrak{F}_{T}^{X}
\end{equation*}
Somit ist auch $A^{C} \in \mathfrak{F}_{T}^{X}$. Seien nun $A_{1},A_{2},... \in \mathfrak{F}_{T}^{X}$. Dann gilt für alle $n \in \mathbb{N}_{0}$
\begin{equation*}
(\bigcup_{i=1}^{\infty} A_{i}) \cap \lbrace T = n \rbrace = (\bigcup_{i=1}^{\infty} \underbrace { A_{i} \cap \lbrace T = n \rbrace}_{ \in \mathfrak{F}_{T}^{X}}) \in \mathfrak{F}_{T}^{X}
\end{equation*}
Also ist auch $\bigcup_{i=1}^{\infty} A_{i} \in \mathfrak{F}_{T}^{X}$.