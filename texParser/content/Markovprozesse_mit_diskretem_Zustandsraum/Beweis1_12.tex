Das Ziel ist es den Satz von Daniell-Kolmogorov anzuwenden.
\\
\\
\underline{Schritt 1} Definiere die Mengenfunktion $\mathbb{Q}_{\lbrace 0,1,...,n \rbrace}: \varepsilon^{ \otimes (n+1)} \to [0, \infty]$ durch 
\begin{equation*}
\mathbb{Q}_{\lbrace 0,1,...,n \rbrace}[\lbrace x_{0},x_{1},..,x_{n} \rbrace] := \nu(x_{0})p(x_{0},x_{1}) \cdot...\cdot p(x_{n-1},x_{n}), \: x_{0},...,x_{n} \in E
\end{equation*}
\dashuline{zu zeigen}:  $\mathbb{Q}_{\lbrace 0,1,...,n \rbrace}$ ist ein Wahrscheinlichkeitsmaß
\\
\\
Offensichtlich ist  $\mathbb{Q}_{\lbrace 0,1,...,n \rbrace}[\emptyset] = 0$ und für alle $A_{1},A_{2},... \in \varepsilon^{ \otimes (n+1)}$ disjunkt gilt
\begin{equation*}
\mathbb{Q}_{\lbrace 0,1,...,n \rbrace}[\cup_{i=1}^{\infty}A_{i}] = \sum _{x_{0},...,x_{n} \in E} \mathbbm{1}_{\bigcup_{i=1}^{\infty}A_{i}}(x_{0},...,x_{n}) \cdot \nu(x_{0})p(x_{0},x_{1}) \cdot ... \cdot p(x_{n-1},x_{n})
\end{equation*}
\begin{equation*}
\stackrel{A_{i} \mathrm{\: disjunkt}}{=} \sum_{i=1}^{\infty} \sum _{x_{0},...,x_{n} \in E} \mathbbm{1}_{A_{i}}(x_{0},...,x_{n}) \cdot \nu(x_{0})p(x_{0},x_{1}) \cdot ... \cdot p(x_{n-1},x_{n})
\end{equation*}
\begin{equation*}
= \sum_{i=1}^{\infty} \mathbb{Q}_{\lbrace 0,1,...,n \rbrace}[A_{i}]
\end{equation*}
Zudem gilt, da P eine stochastische Matrix und $\nu$ ein Wahrscheinlichkeitsvektor ist 
\begin{equation*}
\sum_{x_{0},...,x_{n} \in E} \mathbb{Q}_{\lbrace 0,1,...,n \rbrace}[\lbrace x_{0},...,x_{n} \rbrace] = \sum_{x_{0} \in E} \nu(x_{0}) \sum_{x_{1} \in E} p(x_{0},x_{1}) \: ... \sum_{x_{n} \in E}p(x_{n-1},x_{n}) = 1
\end{equation*}
\dashuline{zu zeigen}: $\mathbb{Q}_{\lbrace 0,1,...,n \rbrace} = \mathbb{Q}_{\lbrace 0,1,...,n+1 \rbrace} \circ ({\pi_{\lbrace 0,...,n \rbrace}}^{\lbrace 0,...,n+1 \rbrace})^{-1} \qquad \forall n \in \mathbb{N}_{0}$
\\
\\
Für $A \in \varepsilon^{ \otimes (n+1)}$ gilt

\begin{equation*}
\mathbb{Q}_{\lbrace 0,1,...,n+1 \rbrace}[({\pi_{\lbrace 0,...,n \rbrace}}^{\lbrace 0,...,n+1 \rbrace})^{-1}(A)]
\end{equation*}
\begin{equation*}
= \sum_{x_{0},...,x_{n} \in A} \nu(x_{0})p(x_{0},x_{1}) \cdot ... \cdot p(x_{n-1},x_{n}) \sum_{x_{n+1} \in E} p(x_{n},x_{n+1}) = \mathbb{Q}_{\lbrace 0,1,...,n \rbrace}[A] 
\end{equation*}
Folglich ist die Familie $\lbrace \mathbb{Q}_{\lbrace 0,1,...,n \rbrace} \: : \: n\in \mathbb{N}_{0}\rbrace$ konsistent. Aus Satz $\ref{Existenzsatz von Daniel und Kolmogorov}$ folgt somit die Existenz genau eines Wahrscheinlichkeitsmaßes $\mathbb{Q}$ auf $(E^{\mathbb{N}_{0}},\varepsilon^{ \otimes \mathbb{N}{0}})$ mit
\begin{equation*}
\mathbb{Q}_{\lbrace 0,1,...,n \rbrace} = \mathbb{Q} \circ {\pi_{\lbrace 0,1,...,n \rbrace}}^{-1}
\end{equation*}
\underline{Schritt 2} Sei $(X_{n})_{n \in \mathbb{N}_{0}}$ ein stochastischer Prozess auf $(\Omega, \mathfrak{F}, \mathbb{P})$ mit Zustandsraum E. Definiere $\mathbb{P}_{X} = \mathbb{P} \circ X^{-1} := \mathbb{Q}$
\\
\\
\dashuline{zu zeigen}: $(X_{n})_{n \in \mathbb{N}_{0}}$ ist eine $(\nu,P)$-Markovkette
\\
\\
Aus Lemma $\ref{Teilmengen Lemma}$ (i) folgt zunächst einmal, dass
\begin{equation*}
\mathbb{P}[X_{0} = x_{0},...,X_{n} = x_{n}] = \mathbb{P}_{X_{\lbrace 0,...,n \rbrace}}[\lbrace x_{0},...,x_{n}  \rbrace] = (\mathbb{P}_{X} \circ {\pi_{\lbrace 0,1,...,n \rbrace}}^{-1})[\lbrace x_{0},...,x_{n}  \rbrace]
\end{equation*}
\begin{equation*}
= (\mathbb{Q} \circ {\pi_{\lbrace 0,1,...,n \rbrace}}^{-1})[\lbrace x_{0},...,x_{n}  \rbrace]
\end{equation*}
\begin{equation*}
= \mathbb{Q}_{\lbrace 0,1,...,n \rbrace}
\end{equation*}
\begin{equation*}
= \nu(x_{0})p(x_{0},x_{1}) \cdot ... \cdot p(x_{n-1},x_{n})
\end{equation*}
Damit folgt die Behauptung aus dem Satz $\ref{Besitzen Markovketten die Markoveigenschaft}$ (i).