\label{Besitzen Markovketten die Markoveigenschaft}
Sei ${(X_{n})}_{n \in \mathbb{N}_{0}}$ eine Folge von E-wertigen Zufallsvariablen auf $(\Omega, \mathfrak{F}, \mathbb{P})$, $\nu$ eine Verteilung auf E und $P = (p(x,y))_{x,y \in E}$ eine stochastische Matrix.
\\
\\
(i) ${(X_{n})}_{n \in \mathbb{N}_{0}}$ ist genau dann eine $(\nu,P)$-Markovkette, wenn für alle $n \in \mathbb{N}_{0}$  und $x_{0},...,x_{n} \in E$ gilt
\begin{equation*}
\mathbb{P}[X_{0} = x_{0},...,X_{n} = x_{n}] = \nu(x_{0}) p(x_{0},x_{1}) \cdot ...\cdot p(x_{n-1},x_{n}).
\end{equation*}
\\
\\
(ii) Ist ${(X_{n})}_{n \in \mathbb{N}_{0}}$ eine $(\nu,P)$-Markovkette, so gilt
\begin{equation*}
\mathbb{P}[X_{n+1} = y \: | \: X_{n} = x] = p(x,y)
\end{equation*} 
für alle $n \in \mathbb{N}_{0}$ und $x,y \in E$ mit $P[X_{n} = x]>0$.